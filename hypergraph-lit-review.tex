\subsection{Hypergraph Representations in Semantic Web}
The Semantic Web standards, primarily RDF and OWL, are based on directed binary graphs where knowledge is expressed as triples connecting a subject to an object via a predicate~\parencite{W3C_RDF11}.  While this simplicity facilitates scalability and graph traversal, it complicates the modeling of complex, high-dimensional systems~\parencite{antelmi2023survey}.  Many systems engineering domains are naturally described by hypergraphs, where a single relation (hyperedge) connects an arbitrary set of nodes.  UDDL requires treating participant paths of associations as first-class citizens when modeling, enabling the definition of semantic relationships between \textit{paths} rather than merely between Entities and Observables~\parencite{OpenGroup2023UDDL}, requiring a hypergraph representation.

\subsubsection{Patterns for Hypergraph Representation}
Since OWL does not natively support $ N $-ary hyperedges, various design patterns have emerged to map hypergraphs into binary relations.  The $ N $-ary Relation Pattern (or Reification) is the most established method for transforming a hyperedge into a distinct individual of a specific class ~\parencite{noy2006defining}.  For example, a `Flight` relationship connecting a `Pilot`, `Aircraft`, and `Route` is modeled as an instance of a `Flight` class with binary properties pointing to each participant.  While this captures the structural semantics of a hyperedge, it inflates the graph size and obscures direct traversal, complicating path-based queries and reasoning~\parencite{hernandez2015reifying}.

Approaches such as Singleton Properties ~\parencite{nguyen2014singleton} and Named Graphs ~\parencite{carroll2005named} allow for attaching metadata to specific relationship instances or identifying sets of triples with a URI.  However, these approaches often lead to high schema volatility and rely on implementation-specific semantics that exist outside the standard logical model of OWL, limiting their utility for automated reasoning about internal path structures and compatibility with standard tooling~\parencite{schneider2014rdf}.

RDF-star is a recent approach that allows triples to be the subject or object of other triples, providing a compact syntax for statements about statements~\parencite{hartig2014foundations}.  While efficient for annotating single hops, it does not natively provide a construct for ordered sequences of arbitrary length, which is a core requirement for UDDL Participant Paths.

To model relationships \textit{between} paths (e.g., ``Path A is associated with Path B''), the path must be reified as a structural entity.  UDDL proposes a Linked List Pattern, a formalized metamodel for defining paths as structural chains.  An \texttt{Association} is composed of \texttt{Participant} elements defined by a \texttt{path} attribute, itself a recursive composition of \texttt{PathNode} elements~\parencite{OpenGroup2023UDDL}.  There are domain-specific ontology extensions that provide similar structures to UDDL's linked list pattern.  The path association requirement mirrors patterns in geospatial ontologies where ``Semantic Trajectories'' aggregate sequences of segments to model inter-path interactions~\parencite{hu2013semantic}, and bioinformatics, where the BioPAX ontology models signaling pathways as hierarchical hypergraphs to define ``crosstalk'' between distinct paths~\parencite{demir2010biopax, fabregat2018reactome}.  However, these domain-specific extensions are not portable across standard tooling and assume domain-specifics rather than allowing a modeler to model arbitrary domains and elements of those domains.

The mapping methodology presented in \nameref{sec:uddl2owl_mapping} builds upon these patterns, specifically utilizing a combination of the $ N $-ary Relation pattern (Reification) for Associations and OWL's Property Chain Axiom to promote UDDL's path-centric logic into the OWL 2 DL framework.
