% =================================================================
% INCOSE Conference LaTeX Template V1.2 (Release Date: November 4th, 2025)
% Copyright (c) 2025 INCOSE
% 
% This template is provided for use in preparing manuscripts for
% INCOSE conferences. You may use, modify, and 
% distribute this template for academic and professional purposes.
% 
% This template is provided "as is" without warranty of any kind.
% The author(s) disclaim all warranties, express or implied,
% including but not limited to warranties of merchantability and
% fitness for a particular purpose.

% =================================================================

\documentclass[11pt,letterpaper]{article} % Remove this line if using A4 size.
%\documentclass[11pt,a4]{article} % A4 is also accepted and is typically used for non-US submissions.

% ---------- Core layout ----------
\usepackage[
  letterpaper,
  left=0.6in,right=0.6in,top=0.6in,bottom=0.6in,
  headheight=85pt,headsep=-45pt
]{geometry}
\raggedbottom
\usepackage{graphicx}
\graphicspath{{./}{figures/}}
\usepackage{float}
\usepackage{amsmath}
\usepackage[table]{xcolor}
\usepackage{tikz}

% ---------- Captions & float spacing ----------
\usepackage{caption}
\captionsetup{
  font={sf,bf,footnotesize},
  labelfont={sf,bf,footnotesize},
  justification=centering,
  labelsep=period,
  hypcap=false
}
\setlength{\textfloatsep}{8pt}
\setlength{\floatsep}{6pt}
\setlength{\intextsep}{8pt}
\setlength{\abovecaptionskip}{12pt}
\setlength{\belowcaptionskip}{1pt}

% ---------- Tables / lists ----------
\usepackage{booktabs}
\usepackage{array}
\usepackage{tabularx}
\usepackage{enumitem}
\setlist{nosep}
\definecolor{tableheader}{HTML}{D9D9D9}
\newcolumntype{P}[1]{>{\sffamily\centering\arraybackslash}p{#1}}
\newcolumntype{Y}{>{\sffamily\centering\arraybackslash}X}
\newcolumntype{A}[1]{>{\raggedright\arraybackslash}p{#1}}
\newcolumntype{M}[1]{>{\raggedright\arraybackslash}m{#1}}

% ---------- Fonts ----------
\usepackage[T1]{fontenc}
\usepackage[utf8]{inputenc}
\usepackage{PTSerif}
\usepackage[scaled]{helvet}
\renewcommand{\sfdefault}{phv}
\newcommand{\headingfont}{\sffamily}

% ---------- Headings ----------
\usepackage{titlesec}
\setcounter{secnumdepth}{1}

% Heading 1
\titleformat{\section}
  {\headingfont\bfseries\raggedright\fontsize{18pt}{18pt}\selectfont}{}{0.75em}{}
% Heading 2
\titleformat{\subsection}
  {\headingfont\bfseries\raggedright\fontsize{15pt}{16pt}\selectfont}{}{0.75em}{}
% Heading 3
\titleformat{\subsubsection}
  {\headingfont\bfseries\raggedright\fontsize{12pt}{14pt}\selectfont}{}{0.75em}{}

% Heading spacing
\titlespacing*{\section}{0pt}{7pt}{6pt}
\titlespacing*{\subsection}{0pt}{7pt}{6pt}
\titlespacing*{\subsubsection}{0pt}{7pt}{6pt}

\newcommand{\miniheading}[1]{%
  \par\noindent{\headingfont\bfseries\fontsize{12pt}{14pt}\selectfont #1}\par\vspace{4pt}%
}

% ---------- Paragraphing ----------
\setlength{\parindent}{0pt}
\setlength{\parskip}{6pt plus 1pt minus 1pt}

% ---------- Page numbers ----------
\usepackage{fancyhdr}
\pagestyle{fancy}
\fancyhf{}
\fancyhfoffset[R]{18pt}
\setlength{\footskip}{18pt}
\fancyfoot[R]{\sffamily\bfseries\footnotesize \thepage}
\renewcommand{\headrulewidth}{0pt}
\renewcommand{\footrulewidth}{0pt}

% INCOSE logo
\fancypagestyle{firstpage}{
  \fancyhf{}
  \fancyhfoffset[R]{18pt}
  \fancyhead[R]{%
    \smash{\raisebox{0pt}[0pt][0pt]{
      \begingroup\setlength{\fboxsep}{20pt}
        \colorbox{white}{\includegraphics[height=0.6in]{template-images/incose-logo.jpg}}%
      \endgroup
    }}
  }
  \fancyfoot[R]{\sffamily\bfseries\footnotesize \thepage}
  \renewcommand{\headrulewidth}{0pt}
  \renewcommand{\footrulewidth}{0pt}
}
% ---------- Title Formatting ----------
\makeatletter
\@ifundefined{theauthor}{}{\let\theauthor\relax}
\makeatother
\usepackage{titling}
\pretitle{\headingfont\bfseries\fontsize{24pt}{26pt}\selectfont\raggedright}
\posttitle{\par\vspace{-.3in}}
\preauthor{}\postauthor{}
\author{\mbox{}}
\date{}
\setlength{\droptitle}{-3.2\baselineskip}

% ---------- Author cards ----------
\newcommand{\authorcard}[5]{%
  {\headingfont\bfseries\fontsize{12pt}{14pt}\selectfont #1}\par
  {\headingfont\bfseries\fontsize{12pt}{14pt}\selectfont #2}\par
  {\headingfont\bfseries\fontsize{12pt}{14pt}\selectfont #3}\par
  {\headingfont\bfseries\fontsize{12pt}{14pt}\selectfont #4}\par
  {\headingfont\bfseries\fontsize{12pt}{14pt}\selectfont #5}\par
}

% ---------- Biography photo placeholder and entry ----------

\makeatletter
\newcommand{\authorpic}[1]{%
    \includegraphics[width=0.6in,height=0.6in,keepaspectratio,clip]{#1}%
}
\makeatother

\newcommand{\authorbioentry}[3]{%
  \noindent\begin{tabular}{@{}m{0.5in} M{\dimexpr\columnwidth-0.5in\relax}@{}}
    \authorpic{#1} & \textbf{#2}\par #3
  \end{tabular}\par\medskip
}

% ---------- Safe figure include ----------
\makeatletter
\newcommand{\colfig}[2][]{%
  \IfFileExists{#2}{\includegraphics[width=\linewidth,#1]{#2}}{%
    \fbox{\parbox[b][1.5in][c]{\linewidth}{\centering \textit{Missing figure: }#2}}}%
}
\makeatother

% ---------- References: APA via biblatex/biber ----------
\let\theauthor\relax
\usepackage{csquotes}
\usepackage[style=apa,backend=biber]{biblatex}
\addbibresource{references.bib}

% ---------- Highlight callouts ----------
\usepackage{changepage}
\newenvironment{highlight}[1][0.25in]{%
  \begin{adjustwidth}{#1}{#1}\itshape}{\end{adjustwidth}}

% ---------- Two-column setup ----------
\usepackage{multicol}
\setlength{\columnsep}{18pt}

% ---------- Hyperlinks ----------
\usepackage[hidelinks]{hyperref}

%Additional packages
%\usepackage[utf8]{inputenc}
%\usepackage[T1]{fontenc}
\usepackage{graphicx}  % For including PNG figures
\usepackage{float}     % For better figure placement
\usepackage{amsmath}   % For mathematical equations
\usepackage{amsfonts}
\usepackage{amssymb}
\usepackage{hyperref}  % For hyperlinks
\usepackage{url}       % For URLs in bibliography
\usepackage{authblk}   % For author affiliations
\usepackage{longtable} % For multi-page tables
\usepackage{siunitx}   % For SI units


% =========================
% ===== Title & Authors ===
% =========================
\title{Aligning the Universal Domain Description Language (UDDL) with Web Ontology Language (OWL)}

\begin{document}
\maketitle
\thispagestyle{firstpage}

% ---- Authors ----
% ---- For the initial paper submission, do not include any author information. For the final paper submission, format author information as shown below. ------------------

\noindent
\begin{tabular*}{\textwidth}{@{\extracolsep{\fill}} A{0.32\textwidth} A{0.32\textwidth} A{0.32\textwidth}}
  \authorcard{Author One}{Organization}{Street Address}{City, Province, Postal}{author.one@email.com} &
  \authorcard{Author Two}{Organization}{Street Address}{City, Province, Postal}{author.two@email.com} &
  \authorcard{Author Three}{Organization}{Street Address}{City, Province, Postal}{author.three@email.com} \\
  \multicolumn{3}{@{}c@{}}{\rule{0pt}{0.9\baselineskip}} \\[-0.2\baselineskip]%\multicolumn{3}{@{}c@{}}{\rule{0pt}{0.9\baselineskip}} \\[-0.2\baselineskip]
  \authorcard{Author Four}{Organization}{Street Address}{City, Province, Postal}{author.four@email.com} &
  \authorcard{Author Five}{Organization}{Street Address}{City, Province, Postal}{author.five@email.com} &
  \authorcard{Author Six}{Organization}{Street Address}{City, Province, Postal}{author.six@email.com}%\authorcard{Author Six}{Organization}{Street Address}{City, Province, Postal}{author.six@email.com}
\end{tabular*}
\addvspace{.75in}

% ---- Two columns begin immediately after authors ----
\begin{multicols*}{2}
\raggedcolumns

% ---- Copyright ----
{\headingfont\bfseries\fontsize{8pt}{12pt}\selectfont
Copyright~\textcopyright~ \the\year{} by the author(s). Permission granted to INCOSE to publish and use.}
\\
% =========================
% ===== Abstract/Keywords =
% =========================
\phantomsection
\miniheading{Abstract}
The engineering of complex, software-intensive systems faces a pervasive challenge: data ambiguity.  As disparate systems ranging from avionics to the Industrial Internet of Things attempt to exchange information, the semantic context of that data is often not communicated clearly, leading to unplanned integration effort and interoperability gaps.  The Universal Domain Description Language (UDDL)\textregistered{} addresses this by providing a platform-independent data modeling language rooted in the separation of concerns between Conceptual, Logical, and Platform data models.  However, while UDDL effectively documents data meaning, it does not verify if data modeling  structures and semantics are physically or operationally meaningful.

This paper proposes a formal mapping from UDDL models and semantic data specifications to Web Ontology Language (OWL) and SPARQL that enables automated semantic verification, allowing systems engineers to programmatically detect mismatched and semantically incoherent interface definitions (e.g., a temperature sensor producing mass data) before physical integration.  We present a methodology where UDDL definitions are transformed into OWL classes and axioms, and system elements are instantiated as named individuals to verify data production and usage against interface definition specifications.  We apply this approach to a representative spacecraft interface model, demonstrating that standard reasoning engines can detect semantic contradictions.  Our results show that this hybrid approach preserves the structural and semantic rigor of UDDL data modeling while enabling automated semantic verification.  We conclude that bridging UDDL with the Semantic Web provides systems engineers with powerful new tools to ensure data interoperability and consistency.


\phantomsection
\subsubsection{Keywords}
Provide a few keywords, separated by commas.

% =========================
% ===== Main Content ======
% =========================
\section{Introduction}
\label{section:introduction}

In systems engineering, rigorous definition of data and interfaces is not merely a documentation exercise, but a fundamental requirement for the composition, maintenance, and future reuse of complex systems~\parencite{fusco2020approach, madni2014system, Friedenthal2014, INCOSE2035}.  In traditional systems engineering, data dictionaries and interface control documents (ICDs) have served as the primary means of semantic capture~\parencite{Friedenthal2014, price2013semantic}.  However, these artifacts are often static in nature and reliant on human interpretation~\parencite{Friedenthal2014}.  The Universal Domain Description Language (UDDL)\textregistered{} Technical Specification recognizes that ``[adequately] describing domain data is a common problem in computer-based systems.  Typical approaches involve creating documentation to accompany artifacts \ldots in an attempt to capture the semantics of the data exchanged''~\parencite{OpenGroup2023UDDL}.

Systems engineering practice has shifted in the past decade to emphasize open systems approaches and adoption of Intelligent Digital Twin-enabled systems, where machine-interpretable data is a prerequisite for modern sociotechnical system integration~\parencite{INCOSE2035}.  Strategic initiatives such as the Modular Open Systems Approach (MOSA) directives~\parencite{esper2019modular, zimmerman2019considerations, guertin2025sustained} also drive the need for semantic alignment, particularly for developing complex cyber-physical systems.  To support this, recent Department of Defense (DoD) directives emphasize the formalization of interface documentation through standards like Open Mission Systems (OMS)~\parencite{USAF_OMS}, the Future Airborne Capability Environment (FACE)\textregistered{}~\parencite{OpenGroup2017FACE}, and Sensor Open Systems Architecture (SOSA)\textregistered{}~\parencite{lane2024pyramid}.  However, current MOSA-aligned workflows rely on domain experts to manually judge the validity of semantic specifications.  This manual process is a significant bottleneck in large-scale system integration.  By bridging UDDL with Semantic Web Technologies, we transform interface review from a subjective manual exercise into more objective, machine-interpretable and automatable tasks.

The failure of traditional, document-based approaches to capture data semantics stems from their inability to formally and precisely bind meaning to structure, i.e., to attach semantics to syntax~\parencite{price2013semantic, Friedenthal2014}.  Ambiguity often arises from the heterogeneous nature of documentation across disparate systems, ranging from avionics to the Industrial Internet of Things (IIoT)~\parencite{mzili2025interoperability, noura2019interoperability}.  For example, a variable labeled \texttt{alt} in a ICD may imply ``altitude'', but it may fail to specify a reference frame (e.g., ellipsoidal vs. geoidal), the unit of measure (e.g., meters vs. feet), or the domain (e.g., air operations vs. ground).  While the integration of data semantics is well-studied in the context of Knowledge Graphs (KGs) and ontologies, UDDL data models remain under-studied in systems and interoperability literature.  Nevertheless, UDDL was developed expressly to fill the documentation gap described above, its objective being to ``formally document the meaning of data with the goal of eliminating ambiguity''~\parencite{OpenGroup2023UDDL}.  It achieves this by providing the means to create conceptual building blocks that are then used to construct formal definitions of the meaning of data flowing through interfaces.

While UDDL is effective at defining the contextual meaning of interface data, it does not provide a way to determine whether the semantics expressed by an interface are \textit{meaningful} within a given domain.  In other words, UDDL allows an interface to specify that a data field represents a measurement produced by a specific component, but it does not assess whether that semantic specification makes sense in the context of a particular system architecture and operational context.  For example, an interface could define a data field as representing a measurement of spacecraft mass produced by a temperature sensor.  Although this statement is syntactically valid and expressible using UDDL, it is semantically nonsensical in the context of spacecraft systems engineering, as temperature sensors do not usually produce mass measurements.  UDDL alone cannot detect such domain-level semantic mismatches in interface definitions, instead relying on domain experts to review the interface definitions and make judgments about the validity of the semantic specifications.

In addition, UDDL, being used for data interface specification, does not support reasoning over the data itself that is exchanged via an interface, and therefore cannot determine whether those values are appropriate in a particular situation.  For example, a spacecraft telemetry interface may identify a specific sensor as the source of a temperature measurement, but UDDL cannot determine whether that sensor is actually a temperature sensor, whether it is installed on the spacecraft, or whether it is suitable for the intended operational context.  While UDDL can define what a data field is intended to represent, it cannot assess whether the instance-level data associated with that field is reasonable or consistent with domain expectations.  Addressing these limitations requires additional mechanisms beyond UDDL to assess both the validity of interface definitions and the correctness of the data they convey.

Semantic Web Technologies (SWTs), designed to enable machine-readability and logical inference, offer a potential solution to this limitation within UDDL ~\parencite{berners2023semantic}.  At the core of this stack are the Resource Description Language (RDF) and the Web Ontology Language (OWL).  RDF provides a means to create graph-based data models where knowledge is encoded as tuples (Subject, Predicate, Object), offering high flexibility and linkability~\parencite{W3C_RDF11}.  OWL extends RDF with formal logical semantics (Description Logics), enabling systems to define complex relationships, cardinality constraints, and classifications~\parencite{W3C_OWL2}.

Despite development and growing industry use of UDDL and its predecessor data modeling languages since 2013~\parencite{OpenGroup2013FACE}, to our knowledge, there is no existing literature showing interoperability, traceability, or mappings between UDDL data models and OWL or RDF.  The objective of this work is to develop an approach that enables UDDL domain-specific data models to interoperate with SWTs, leading to more efficient integration between systems based on UDDL and SWTs. 

In \nameref{sec:background}, the relevant technologies are described and relevant literature on the subject is reviewed.  The methodology used to approach this problem is described in \nameref{sec:methods}.  The results of this work are presented in \nameref{sec:results} and are analyzed in \nameref{sec:analysis}.  Limitations of this research and potential avenues for future research are discussed in \nameref{sec:discussion} and the paper is concluded in \nameref{sec:conclusion}.

\section{Background} \label{sec:background}

\subsection{UDDL} \label{section:introduction-uddl-overview}
Any system developed against the FACE and SOSA technical standards necessarily use UDDL data models for interface specification.  UDDL utilizes a multi-level modeling approach summarized in Table \ref{tab:uddl_levels}~\parencite{OpenGroup2023UDDL}.

\begin{table*}[t]
    \centering
    \begin{tabular}{|p{0.25\linewidth}|p{0.35\linewidth}|p{0.3\linewidth}|}
        \hline
        \textbf{Model Level} & \textbf{Description} & \textbf{Semantic Focus} \\
        \hline
        Conceptual Data Model (CDM) & Defines the complete semantics of all Entities and Associations. & Concepts and relationships.  Entities and Associations are typed with Observables (traits which can be observed but not further characterized). \\
        \hline
        Logical Data Model (LDM) & Refines Entities/Associations with Measurements. & Quantification.  Further characterizes Observables by adding frames of reference, units, and value types (e.g., Integer, Real). \\
        \hline
        Platform Data Model (PDM) & Refines elements with Platform Data Types. & Implementation.  Additional characterization of logical measurements to include bit-level precision and language bindings in terms of IDL. \\
        \hline
    \end{tabular}
    \caption{UDDL Multi-level Modeling Approach.}
    \label{tab:uddl_levels}
\end{table*}

This paper will focus primarily on the Conceptual Data Model (CDM), as the meaning (i.e., the semantics) of the data are completely described by conceptual elements and their relationships.  We will draw upon lower-level LDM and PDM concepts to justify design decisions to ensure the mappings are compatible with lower-level refinements.  Therefore, we will review only the conceptual meta-model elements and their relationships.

Table \ref{tab:cdm_elements} informally summarizes the construction of relevant conceptual meta-model elements.  All elements are assumed to have a globally unique name as well as an optional textual description field, enfoced via Object Constraint Language (OCL) constraints.  A Conceptual Data Model (CDM) is a top-level package that collects and organizes all conceptual model elements.  CDMs can be nested to create hierarchies or other organizational structures and multiple CDMs may be included in a single root-level Data Model.  A Domain collects a set of Basis Entities, relating to well understood concepts by practitioners within a particular domain~\parencite{OpenGroup2023UDDL}.  Basis Entities represent unique concepts that are not expressible simply through model structure, e.g., the concepts of ``Minimum'' and ``Maximum''~\parencite{sdm_v3_1_15}.  Observables are characteristics that can be observed or measured, e.g., a Position, Orientation, Identifier, and are used to ``type'' Entities.  Notice the examples for Basis Entity and Observable in table \ref{tab:cdm_elements} -- a ``Position'' is something that can be measured and is therefore an Observables while ``Place'' is a concept and not meaningfully measurable and is therefore a Basis Entity (i.e., you can not measure ``Place'' directly, but you could measure the Position of a Place, or the Extent of a Place, Position and Extent being Observables).  Entities represent a domain concept in terms of its Observables and other composed conceptual Entities.  Associations are themselves Entities but which also include two or more Participants that can associate model paths terminating in either an Entity or Observable.  The Participants in the Association are characterized through Participant Paths which are a sequence of nodes that specify an arbitrary path through the network of related Entities and Associations and follow traversal rules, namely that Compositions can be traversed in the direction of the composition and Participants of Associations can be traversed bidirectionally.

Associations of Participant Paths require a hypergraph representation of a UDDL data model, i.e., relationships are expressed not just between nodes of the graph, but between higher-level graph structures as well.

\begin{table*}[t]
    \centering
    \begin{tabular}{|p{0.20\linewidth}|p{0.40\linewidth}|p{0.40\linewidth}|}
        \hline
        \textbf{Element Name} & \textbf{Construction} & \textbf{Example} \\
        \hline
    
        Conceptual Data Model & Package conceptual model elements in a single collection. & \texttt{Observables} (a collection of all Observable data model elements). \\
        \hline
        Domain & A collection of Basis Entities. & \texttt{Sensing}; includes the \texttt{Object} Basis Entity~\parencite{sdm_v3_1_15}. \\
        \hline
        Basis Entity & Unique axiomatic domain concept. & \texttt{Place}~\parencite{sdm_v3_1_15} \\
        \hline
        Observable & Unique trait which can be observed but not further characterized. & \texttt{Position}~\parencite{sdm_v3_1_15} \\
        \hline
        Entity & A domain concept characterized by a set of composed Observables and/or other Entities.  Can Specialize another Entity or Domain Entity. & \texttt{Person}; composes \texttt{Identifier} and \texttt{Position} Observables. \\
        \hline
        Association & A domain concept that is expressible only through the relation of two or more Participants (Entities and/or Observables).  Associations may also compose a set of Entities and/or Observables and Specialize another Entity or Domain Entity. & \texttt{Ownership}; composes \texttt{Duration} and associates \texttt{Person}, \texttt{Car} Entities through \texttt{owner}, \texttt{asset} Participant roles. \\
        \hline
    \end{tabular}
    \caption{Core UDDL Conceptual Meta-Model Elements.} \label{tab:cdm_elements}
\end{table*}

\begin{table*}[t]
    \centering
    \begin{tabular}{|p{0.15\linewidth}|p{0.55\linewidth}|p{0.3\linewidth}|}
        \hline
        \textbf{Relation} & \textbf{Construction} & \textbf{Example} \\
        \hline
        Composition & Aggregation of conceptual elements (Entities/Observables) into a parent Entity.  Includes multiplicity specifications (upper and lower bounds).  Can be traversed by Participant Paths in the direction of the composition, i.e., from parent to child. & \texttt{Car} composes at most $ 4 $ \texttt{Wheels}. \\
        \hline
        Participant & Role-based link in an Association pointing to an Entity/Observable via a path.  Includes source/target multiplicities (upper and lower bounds).  Can be traversed bidirectionally by Participant Paths. & In the Association \texttt{Teaching}, the participant Entity \texttt{Person} has a single role of \texttt{instructor} and a second participant \texttt{Person} has role of \texttt{student}, potentially many. \\
        \hline
        Generalization & ``is-a'' relationship where a specialized Entity declares it is a specialization of another Entity.  Cannot be traversed by Participant Paths. & \texttt{FixedWingAircraft} specializes \texttt{Aircraft}. \\
        \hline
        Realization & Refinement relationship binding abstract concepts to more specific definitions across model levels (CDM $\to$ LDM).  Cannot be traversed by Participant Paths (no meaning by doing this). & \texttt{AltitudeMeters} (LDM) realizes \texttt{Altitude} (CDM). \\
        \hline
    \end{tabular}
    \caption{UDDL Conceptual Relationships.} \label{tab:cdm_relationships}
\end{table*}

Table \ref{tab:cdm_relationships} summarizes how relationships between core UDDL Conceptual Meta-Model Elements are constructed.  Composition relationships include optional multiplicity specifications to indicate the minimum and maximum number of elements that can be composed ($ -1 $ indicates unbounded).  If not specified, the default is one.  Similarly, Participants of Associations include optional source and target multiplicities that are applied when calculating the multiplicities along the Participant Paths.  Participant multiplicities are assumed to be unbounded if not specified.


\begin{figure*}
    \centering
    \includegraphics[width=0.8\textwidth]{figures/entity-association.png}
    \caption{Example of some features of UDDL Entity construction.  Gray boxes are composed Entities and Observables, e.g., \texttt{EntityX} composes \texttt{ObservableA} and \texttt{ObservableB}.  Solid, dotted, and dashed lines represent Participant Paths.  The Participant Path associated with \texttt{participant3} traverses two nodes, through the composed \texttt{EntityZ} of \texttt{EntityY} and terminating on the \texttt{ObservableD} of \texttt{EntityZ}. No multiplicities are specified for compositions and participant paths and should therefore default multiplicities should be assumed. }
    \label{fig:entity-association}
\end{figure*}

Figure \ref{fig:entity-association} provides a summary of some of the features of Entity and Associations construction.  Entities \texttt{EntityX}, \texttt{EntityY}, and \texttt{EntityZ} compose Observables and other Entities (grey boxes).  The single \texttt{Association} associates Entities and Observables through three different Participant Paths represented by solid, dotted, and dashed lines.  \texttt{participant1} simply associates the entire \texttt{EntityX}.  \texttt{participant2} specifically associates the \texttt{ObservableC} of \texttt{EntityY}.  \texttt{participant3} associates the \texttt{ObservableD} of \texttt{EntityZ} through the composed \texttt{EntityZ} within \texttt{EntityY}, demonstrating a non-trivial multi-node Participant Path.

\subsection{UDDL Query Construction} \label{section:query_construction}

The UDDL Query grammar and associated construction rules defines a SQL-like language for specification of data semantics~\parencite{OpenGroup2023UDDL}.  Query specifications can be constructed at all three levels of the UDDL meta-model.  The only differences in expression of query specifications at the various levels are the meta-types of Entities and Associations (CDM, LDM, PDM; CDM used for conceptual query specifications, LDM for logical query specifications, etc.) and qualifiers that may be used (e.g., in a \texttt{WHERE} clause).  \texttt{WHERE} clauses and other qualifiers specify data filtering criteria and side-effects on generated syntax when combined with a FACE Template specification~\parencite{OpenGroup2017FACE} but do not contribute to the semantics of the query.  We therefore need not consider mapping of \texttt{WHERE} clauses and other qualifiers, including sub-query specifications.  Entities and Associations will map cleanly across realizations between levels of the UDDL meta-model up to down-selection.  Considering this along with the fact that Generalization relationships must also be realized across refinement levels (up to down-selection), we will without loss of generality consider only conceptual query specifications and \texttt{SELECT} and \texttt{JOIN} query language elements.

\subsection{Semantic Web Technologies} \label{section:introduction-swts-overview}

The SWT stack provides a layered architecture for representing, querying, and reasoning over machine-interpretable knowledge. Each layer builds on the capabilities of the layers below it, enabling progressively richer forms of semantic expression and analysis. In the context of this work, SWTs provide the formal foundations needed to represent the semantics implied by interface definitions and to assess their consistency with domain expectations.  Table \ref{tab:swt_overview} summarizes the core technologies relevant to this mapping.

\begin{table*}[t]
    \centering
    \begin{tabular}{|p{0.12\linewidth}|p{0.88\linewidth}|}
        \hline
        \textbf{Technology} & \textbf{Description} \\
        \hline
        RDF & The Resource Description Framework provides a graph-based data model where knowledge is represented as triples (Subject, Predicate, Object), serving as the foundation for data interchange. \\
        \hline
        OWL & The Web Ontology Language extends RDF with formal Description Logic semantics, enabling the definition of complex class hierarchies, restrictions, and inferential rules. \\
        \hline
        SPARQL & The standard query language for RDF, allowing for the retrieval and manipulation of data stored in RDF format. \\
        \hline
        SHACL & The Shapes Constraint Language provides a mechanism to validate RDF graphs against a set of conditions (shapes), ensuring structural conformance. \\
        \hline
    \end{tabular}
    \caption{Overview of Semantic Web Technologies.}
    \label{tab:swt_overview}
\end{table*}

At the base of the stack, RDF provides a graph-based data model in which knowledge is represented as triples consisting of a subject, predicate, and object. RDF offers a flexible and extensible representation that is well suited to integrating heterogeneous data sources and expressing relationships between entities. When interface data is mapped into RDF, individual data values can be represented as nodes within a graph and linked to other entities through explicitly defined relationships, preserving the contextual structure implied by the interface semantics.

Building on RDF, the OWL introduces formal semantics, enabling the definition of domain concepts, relationships, and constraints in a machine-interpretable form.  In particular, we use the Description Logic profile of OWL (OWL2DL), which represents a decidable subset of First-Order Logic (FOL).  OWL supports the specification of class hierarchies, property restrictions, and logical conditions that capture domain knowledge beyond what can be expressed through data structure alone.  These semantics enable automated reasoning, allowing implicit knowledge to be inferred from explicitly stated facts.  In the context of interface analysis, OWL provides a means to determine whether the semantics implied by an interface definition or its associated data are consistent with domain-level expectations.

The SPARQL Protocol and RDF Query Language (SPARQL) complements RDF and OWL by providing a standardized query language for retrieving and manipulating RDF data. SPARQL enables selective access to graph data based on structural patterns and semantic relationships, supporting both exploratory analysis and downstream processing. In an interface-driven workflow, SPARQL can be used to extract relevant subsets of mapped interface data, identify entities participating in specific relationships, or retrieve inferred knowledge produced by reasoning processes.

Finally, the Shapes Constraint Language (SHACL) provides a mechanism for validating RDF graphs against a set of explicit structural and value constraints. While OWL focuses on logical inference under open-world assumptions, SHACL enables closed-world validation of data completeness and structural conformance. Although SHACL is not the primary focus of this work, it represents an additional capability within the SWT stack that can be used to complement reasoning-based analysis when strict data validation is required.

\subsection{Closed-World and Open-World Assumptions} \label{section:theory-cwa-owa}

UDDL is used to construct domain-specific data models (DSDMs), functioning as a descriptive language to talk about all the data flowing through system interfaces.  Any concept not able to be captured using this data model definitionally cannot be understood by any interfaced component in the system if the interfaces are fully characterized using a common data model. SWTs generally adopt an open-world assumption (OWA), in which the absence of information does not imply falsity. Under OWA, a knowledge base is assumed to be inherently incomplete, and additional information may always exist elsewhere. For example, if a knowledge base contains information about a \texttt{Person} but does not state whether that person has any siblings, a reasoning engine cannot conclude that the person has no siblings. It can only conclude that the presence or absence of siblings is unknown, and further facts may be asserted later without contradiction.

In contrast, a closed-world assumption (CWA) treats a knowledge base as complete with respect to the facts it represents: information that is not explicitly stated is assumed to be false. Using the same example, if a closed-world system represents a \texttt{Person} and does not list any brothers, the system concludes that the person has no brothers. Any siblings not represented are assumed not to exist within the modeled world. Closed-world reasoning is common in databases and many engineering modeling environments, where completeness and determinism are required for decision-making.

The distinction between OWA and CWA has important implications for how knowledge is represented and interpreted. Open-world reasoning supports distributed and incremental knowledge integration, while closed-world reasoning enables definitive conclusions based on explicitly stated information. Understanding this distinction is essential when applying Semantic Web Technologies or integrating them with systems that rely on closed-world assumptions.

Because UDDL relies on a closed-world view of completeness while Semantic Web Technologies adopt an open-world assumption, representing UDDL concepts using SWTs requires the introduction of ``closure axioms''. These axioms explicitly restrict which properties or relationships are considered complete for a given entity, thereby simulating the structural rigor and completeness guarantees of closed-world representations within an open-world reasoning framework.

% This will also complicate our mapping from SWTs to UDDL and forces us to admit that not all ontological structures are mappable back to UDDL.

\subsection{Hypergraph Representations in Semantic Web}
The Semantic Web standards, primarily RDF and OWL, are foundational upon directed binary graphs where knowledge is expressed as triples connecting a subject to an object via a predicate~\parencite{W3C_RDF11}.  While this simplicity facilitates scalability and graph traversal, it becomes difficult to model complex, high-dimensional systems~\parencite{antelmi2023survey}.  Many systems engineering domains are naturally described by hypergraphs, where a single relation (hyperedge) connects an arbitrary set of nodes.  UDDL reqiures treating participant paths of associations first-class citizens when modeling, enabling the definition of semantic relationships \textit{between} paths rather than merely between Entities and Observables~\parencite{OpenGroup2023UDDL}, requiring a hypergraph representation.

\subsubsection{Patterns for Hypergraph Representation}
Since OWL does not natively support $ n $-ary hyperedges, various design patterns have emerged to map hypergraphs into binary relations.  The $ N $-ary Relation Pattern (or Reification) is the most established method for transforming a hyperedge into a distinct individual of a specific class ~\parencite{noy2006defining}.  For example, a `Flight` relationship connecting a `Pilot`, `Aircraft`, and `Route` is modeled as an instance of a `Flight` class with binary properties pointing to each participant.  While this captures the structural semantics of a hyperedge, it inflates the graph size and obscures direct traversal, complicating path-based queries and reasoning~\parencite{hernandez2015reifying}.

Approaches such as Singleton Properties ~\parencite{nguyen2014singleton} and Named Graphs ~\parencite{carroll2005named} allow for attaching metadata to specific relationship instances or identifying sets of triples with a URI.  However, these approaches often lead to high schema volatility and rely on implementation-specific semantics that exist outside the standard logical model of OWL, limiting their utility for automated reasoning about internal path structures and compatibility with standard tooling~\parencite{schneider2014rdf}.

RDF-star is a recent approach that allows triples to be the subject or object of other triples, providing a compact syntax for statements about statements~\parencite{hartig2014foundations}.  While efficient for annotating single hops, it does not natively provide a construct for ordered sequences of arbitrary length, which is a core requirement for UDDL Participant Paths.

To model relationships \textit{between} paths (e.g., ``Path A is associated to Path B''), the path must be reified as a structural entity.  UDDL proposes a Linked List Pattern, a formalized metamodel for defining paths as structural chains.  An \texttt{Association} is composed of \texttt{Participant} elements defined by a \texttt{path} attribute, itself a recursive composition of \texttt{PathNode} elements~\parencite{OpenGroup2023UDDL}.  There are domain-specific ontology extensions that provide similar structures to UDDL's linked list pattern.  The path association requirement mirrors patterns in geospatial ontologies where ``Semantic Trajectories'' aggregate sequences of segments to model inter-path interactions~\parencite{hu2013semantic}, and bioinformatics, where the BioPAX ontology models signaling pathways as hierarchical hypergraphs to define ``crosstalk'' between distinct paths~\parencite{demir2010biopax, fabregat2018reactome}.  However, these domain-specific extensions are not portable across standard tooling and assume domain-specifics rather than allowing a modeler to model arbitrary domains and elements of those domains.

The mapping methodology presented in \nameref{sec:uddl2owl_mapping} builds upon these patterns, specifically utilizing a combination of the $ N $-ary Relation pattern (Reification) for Associations and OWL's Property Chain Axiom to promote UDDL's path-centric logic into the OWL 2 DL framework.

\subsection{UDDL to OWL Mapping} \label{sec:uddl2owl_mapping}

\subsubsection{Containers}

UDDL provides packaging constructs to organize definitions.  We map these containers directly to \texttt{owl:Ontology} resources.  The hierarchy of UDDL models is preserved through \texttt{owl:imports} statements, where a child model imports the ontology generated from its parent or dependent models.  This approach maintains the modularity of the UDDL specification while ensuring that all definitions are available to the reasoner.

\subsubsection{Generalization}

UDDL Generalization is mapped directly to the \texttt{rdfs:subClassOf} property in OWL.  When a UDDL model specifies that element $ A $ \texttt{specializes} element $ B $, the transformation generates an OWL axiom declaring class $ A $ as a subclass of class $ B $.  This preserves the strict set-inclusion semantics of UDDL generalization, ensuring that every instance of the specialized concept is also recognized as an instance of the general concept.  Additionally, the transformation defines top-level subsumption axioms, ensuring that all Entities eventually specialize a root \texttt{Entity} class, maintaining a consistent ontology structure.

\subsubsection{Observable}

UDDL Observables are mapped to the \texttt{owl:Class} construct.  Each Observable declared in the UDDL model becomes a subclass of a common root \texttt{Observable} class.  To capture the distinct nature of different observables, the transformation defines an \texttt{owl:AllDisjointClasses} axiom covering the set of all defined Observables.  This prevents an individual from being inferred as belonging to multiple conflicting observable types simultaneously, enforcing semantic precision.  While \texttt{owl:DatatypeProperty} could theoretically represent simple values, mapping Observables to classes provides greater extensibility for future refinement into logical measurements without altering the fundamental ontological structure. 

\subsubsection{Entity}

UDDL Entities are mapped to the \texttt{owl:Class} construct, consistent with their role as categorization mechanisms.  The conversion process automatically generates a global \texttt{composes} Object Property with a domain of \texttt{Entity} and a range defined as the union of \texttt{Entity} and \texttt{Observable} classes.  Composition relationships are then realized as \texttt{owl:Restriction} axioms on the Entity class.  This approach uses the shared \texttt{composes} property and distinguishes components via the \texttt{owl:onClass} constraint.  Multiplicities are strictly enforced using \texttt{owl:minQualifiedCardinality} and \texttt{owl:maxQualifiedCardinality}, with unbounded compositions represented by the omission of the maximum cardinality constraint.  To facilitate bidirectional graph traversal, an inverse property \texttt{isComposedBy} is also generated.

\subsubsection{Domain}

A UDDL Domain is a collection of Basis Entities that defines a vocabulary for a specific subject area.  We map each UDDL Domain to a distinct \texttt{owl:Ontology}.  This separation ensures that domain concepts are modular and reusable.  Other ontologies can reference these domains via \texttt{owl:imports}, allowing for the composition of complex system models from standard domain definitions.

\subsubsection{Basis Entity}

Basis Entities represent the fundamental axiomatic concepts within a Domain.  We map Basis Entities to \texttt{owl:Class}, identical to the mapping for standard Entities.  While Basis Entities often serve as the roots of taxonomic trees (using \texttt{rdfs:subClassOf} for derived concepts), they possess no special meta-model distinction in OWL.  They are simply classes that provide the grounding for the domain vocabulary.

\subsubsection{Association}

UDDL conceptual Associations are mapped to \texttt{owl:Class}, inheriting from the root \texttt{Entity} class.  Its participants are realized using the global \texttt{associates} Object Property (and its inverse \texttt{isParticipantIn}).  However, unlike simple direct relationships, UDDL participants can be defined by complex paths traversing multiple entities.  The OWL transformation handles this by simplifying the Class-level definition: the \texttt{associates} restriction on the Association class targets the type of the final node in the participant path.  The full semantic rigor of the path is pushed to the instance level, where specific property chains can be verified.  By asserting the full path structure on instances, the model supports validation via \texttt{owl:propertyChainAxiom}, ensuring that the data actually traverses the required structural path defined in the UDDL model.  As UDDL Association participant source and target upper and lower bounds do not define semantics, the bounds of individual nodes in the participant path are excluded from the mapping, instead using the effective bounds for the \texttt{associates} cardinality.

\subsubsection{Query}

The UDDL query language is primarily a Selection and Projection language designed to retrieve specific data semantic definitions by traversing the domain model.  It traverses the model graph through a series of \texttt{JOIN} operations and selects specific Observables.  To enable automated verification, we convert these specifications into SPARQL, the standard query language for the Semantic Web.  This transformation bridges the gap between the relational style of UDDL (tables and joins) and the graph style of SPARQL (nodes and edges) using a two-step process: (1) Path Resolution and (2) Graph Pattern Generation.

\textit{Step 1: Path Resolution (The Alias Map)} \\
First, we analyze the UDDL query to determine exactly how every piece of data is connected to the root entity.  We construct an \textit{Alias Map} ($ \mathcal{M} $), which traces the full navigational path from the \texttt{FROM} entity to every \texttt{JOIN} alias.  Unlike a standard database query plan which executes joins in steps, this map calculates the absolute path for every participant in the query.

\begin{table*}[t]
\centering
\renewcommand{\arraystretch}{1.5}
\begin{tabularx}{\textwidth}{@{}l >{\raggedright\arraybackslash}X >{\raggedright\arraybackslash}X@{}}
    \rowcolor{tableheader}
    \textbf{\headingfont UDDL Component} & \textbf{\headingfont Logic} & \textbf{\headingfont SPARQL Graph Representation} \\
    \addlinespace[4pt]
    \texttt{FROM Entity AS a} & \textbf{Root}: The starting point of the graph traversal. & Root Variable $ ?a $ \\
    \addlinespace[4pt]
    \texttt{JOIN Target AS b ON a.role} & \textbf{Edge}: A traversal from one entity to another via a relationship. & Triple Pattern $ (?a \text{ :role } ?b) $ \\
    \addlinespace[4pt]
    \texttt{SELECT b.attr} & \textbf{Projection}: Identifying the specific data value to retrieve. & Select Variable $ ?b\_attr $ from $ (?b \text{ :attr } ?b\_attr) $ \\
    \addlinespace[4pt]
    \texttt{AND} (in Joins) & \textbf{Convergence}: Requiring an entity to satisfy multiple relationships simultaneously. & Shared Variable $ ?b $ used in multiple patterns (e.g., $ ?a \to ?b $ AND $ ?c \to ?b $) \\
    \bottomrule
\end{tabularx}
\caption{Mapping UDDL Query Components to Semantic Graph Patterns}
\label{table:ast_mapping}
\end{table*}

Crucially, this step resolves complex constraints like ``Diamond Joins'' or \texttt{AND} conditions.  If a UDDL query specifies that an entity $ D $ must be joined to both $ B $ and $ C $ (e.g., \texttt{JOIN D ON D.ref1 = B AND D.ref2 = C}), the Alias Map records two distinct paths reaching the same alias $ D $.  This captures the requirement that $ D $ must be the intersection (or convergence) of these two paths.

\textit{Step 2: Graph Pattern Generation} \\
In the second phase, the resolved paths are converted into a SPARQL query pattern.  Each unique alias in the UDDL query becomes a variable in the SPARQL query (e.g., alias \texttt{sensor} becomes variable \texttt{?sensor}).  Each step in the path becomes a logic statement (a ``triple pattern'') that the data must satisfy.

The logical ``AND'' is enforced implicitly by the structure of the graph pattern.  By using the same variable name for the destination of multiple paths, we constrain the query engine to find only those data instances that satisfy all incoming relationships.  For example, if the map contains paths $ A \to B \to D $ and $ A \to C \to D $, the generated SPARQL will require the variable $ ?d $ to be connected to both $ ?b $ and $ ?c $.  If no such data exists in the system that satisfies this topology, the query returns no results, correctly indicating a semantic mismatch or missing data.

Table \ref{table:ast_mapping} summarizes how the primary components of a UDDL query (Selection, Roots, Joins, and Logic) map to the graph-based concepts of SPARQL.

\section{Methods} \label{sec:methods}

\begin{figure*}[t]
  \centering
  \colfig{figures/ontology-diagram.png}
  \caption{Reference Satellite Ontology}
  \label{fig:satellite_ontology}
\end{figure*}

We generated example Interface Control Documents (ICDs) based on a generic satellite design, provided in the \nameref{sec:Appendix}.  Based on these ICDs, we then created a reference satellite OWL ontology in Prot\'eg\'e and a UDDL data model using the PHENOM\textregistered{} modeling environment, and exported to XMI~\parencite{musen2015protege, phenom}.  The reference satellite ontology captured the structure of the satellite system components and their relationships from a system viewpoint.  The UDDL data model captured the interface semantics using the UDDL Query Language backed by a conceptual Entity and Association data model.  Diagrams of the ontology and data model are shown in Figure \ref{fig:satellite_ontology} and Figure \ref{fig:satellite_data_model}, respectively.  The UDDL data model contains examples of Generalizations, Entities with multiple compositions, Associations with simple and complex participant paths, as well as UDDL Query specifications.  The FACE Shared Data Model (SDM)~\parencite{sdm_v3_1_15} provides examples for Domain, Basis Entity, and Observable model elements.  The set of Queries generated from the example satellite ICDs are shown in Table \ref{tab:query_examples}.



\begin{figure*}[t]
  \centering
  \colfig{figures/uddl-data-model.png}
  \caption{UDDL Data Model}
  \label{fig:satellite_data_model}
\end{figure*}

The UDDL data model was transformed into an OWL ontology using the mapping methodology presented in \nameref{sec:uddl2owl_mapping}.  The conversion was implemented in Python by processing the OWL and UDDL data model XML representations.  The generated ontology was then loaded into Prot\'eg\'e with the HermiT reasoner version 1.4.3 configured~\parencite{glimm2014hermit}.  After loading the ontology, the reasoner was started.  The Prot\'eg\'e environment was used to visualize and execute SPARQL queries against the ontology.  The Snap SPARQL plugin was used to allow queries to be run over inferred elements of the ontology~\parencite{horridge2015snap}.  An ontology alignment was manually generated between the classes in the reference and generated ontologies to enable reasoning across both.

To verify the semantic validity, we created instances for system data observations.  An Observation OWL class was created to allow named individuals representing measurements of Observables in the context implied by the UDDL Query specifications.  We then converted all the UDDL Query specifications into SPARQL queries (shown in Table \ref{tab:query_examples}) and executed them against the generated ontology and individuals.  UDDL Queries were then slightly modified to imply different semantics and run against the original set of individuals.

\section{Results} \label{sec:results}

The generated ontology was loaded into Prot\'eg\'e with the HermiT reasoner configured.  The reasoner verified that there were no logical contradictions.  Table \ref{tab:query_examples} shows the results of the UDDL Query specification conversion to SPARQL queries against the generated ontology.  All SPARQL queries returned the expected sets of individuals indicating that the system contains the expected data observations based on the interfaces specifications.  Any changes to the UDDL Query specifications resulted in SPARQL queries that returned no results indicating the expected semantic inconsistency (i.e., interfaces that require data that is not present in the system).

\begin{table*}[t]
    \centering
    \renewcommand{\arraystretch}{1.5}
    \begin{tabular}{|p{0.95\linewidth}|}
\hline
\textbf{UDDL Query (Solar Panel Monitoring)} \\ 
\hline
\scriptsize\texttt{SELECT solarPanel\_join.identifier AS solar\_panel\_id, \newline
\hspace*{3.5em}solarPanel\_join.healthState AS solar\_panel\_deploy, \newline
\hspace*{3.5em}panel\_join.temperature AS solar\_panel\_temp, \newline
\hspace*{3.5em}solarPanel\_IncidenceAngle\_Sun\_join.angle AS solar\_panel\_incidence, \newline
\hspace*{3.5em}solarPanel\_join.operationalState AS solar\_panel\_ops\_mode \newline
FROM Satellite AS satellite\_from \newline
JOIN SolarPanel\_PartOf\_Satellite AS solarPanel\_PartOf\_Satellite\_join \newline
\hspace*{2.0em}ON solarPanel\_PartOf\_Satellite\_join.assembly = satellite\_from \newline
JOIN SolarPanel AS solarPanel\_join \newline
\hspace*{2.0em}ON solarPanel\_PartOf\_Satellite\_join.part \newline
JOIN TemperatureSensor\_PartOf\_Satellite AS temperatureSensor\_PartOf\_Satellite\_join \newline
\hspace*{2.0em}ON temperatureSensor\_PartOf\_Satellite\_join.assembly = satellite\_from \newline
JOIN TemperatureSensor AS temperatureSensor\_join \newline
\hspace*{2.0em}ON temperatureSensor\_PartOf\_Satellite\_join.part \newline
JOIN TemperatureSensor\_Observe\_SolarPanel AS temperatureSensor\_Observe\_SolarPanel\_join \newline
\hspace*{2.0em}ON temperatureSensor\_Observe\_SolarPanel\_join.observer = temperatureSensor\_join \newline
JOIN SolarPanel AS solarPanel\_joinI \newline
\hspace*{2.0em}ON temperatureSensor\_Observe\_SolarPanel\_join.observed \newline
JOIN Panel\_PartOf\_SolarPanel AS panel\_PartOf\_SolarPanel\_join \newline
\hspace*{2.0em}ON panel\_PartOf\_SolarPanel\_join.assembly = solarPanel\_joinI \newline
JOIN Panel AS panel\_join \newline
\hspace*{2.0em}ON panel\_PartOf\_SolarPanel\_join.part \newline
JOIN SunSenor\_PartOf\_Satellite AS sunSenor\_PartOf\_Satellite\_join \newline
\hspace*{2.0em}ON sunSenor\_PartOf\_Satellite\_join.assembly = satellite\_from \newline
JOIN SunSensor AS sunSensor\_join \newline
\hspace*{2.0em}ON sunSenor\_PartOf\_Satellite\_join.part \newline
JOIN SunSensor\_Observe\_IncidenceAngle AS sunSensor\_Observe\_IncidenceAngle\_join \newline
\hspace*{2.0em}ON sunSensor\_Observe\_IncidenceAngle\_join.observer = sunSensor\_join \newline
JOIN SolarPanel\_IncidenceAngle\_Sun AS solarPanel\_IncidenceAngle\_Sun\_join \newline
\hspace*{2.0em}ON sunSensor\_Observe\_IncidenceAngle\_join.observed} \\ 
\hline
\textbf{SPARQL Translation} \\ 
\hline
\scriptsize\texttt{SELECT ?obs\_0\_identifier ?obs\_1\_healthState ?obs\_2\_temperature ?obs\_3\_angle ?obs\_4\_operationalState \newline
WHERE \{ \newline
\hspace*{1.0em}?obs\_0\_identifier a :Observation . \newline
\hspace*{1.0em}?obs\_0\_identifier :hasObservable :Identifier . \newline
\hspace*{1.0em}?obs\_0\_identifier :hasSatellite :Satellite . \newline
\hspace*{1.0em}?obs\_0\_identifier :hasSatelliteParticipant\_assembly :SolarPanel\_PartOf\_Satellite . \newline
\hspace*{1.0em}?obs\_0\_identifier :hasSolarPanel\_PartOf\_SatelliteComposition\_part :SolarPanel . \newline
\hspace*{1.0em}?obs\_1\_healthState a :Observation . \newline
\hspace*{1.0em}?obs\_1\_healthState :hasObservable :HealthState . \newline
\hspace*{1.0em}?obs\_1\_healthState :hasSatellite :Satellite . \newline
\hspace*{1.0em}?obs\_1\_healthState :hasSatelliteParticipant\_assembly :SolarPanel\_PartOf\_Satellite . \newline
\hspace*{1.0em}?obs\_1\_healthState :hasSolarPanel\_PartOf\_SatelliteComposition\_part :SolarPanel . \newline
\hspace*{1.0em}?obs\_2\_temperature a :Observation . \newline
\hspace*{1.0em}?obs\_2\_temperature :hasObservable :Temperature . \newline
\hspace*{1.0em}?obs\_2\_temperature :hasSatellite :Satellite . \newline
\hspace*{1.0em}?obs\_2\_temperature :hasSatelliteParticipant\_assembly :TemperatureSensor\_PartOf\_Satellite . \newline
\hspace*{1.0em}?obs\_2\_temperature :hasTemperatureSensor\_PartOf\_SatelliteComposition\_part :TemperatureSensor . \newline
\hspace*{1.0em}?obs\_2\_temperature :hasTemperatureSensorParticipant\_observer :TemperatureSensor\_Observe\_SolarPanel . \newline
\hspace*{1.0em}?obs\_2\_temperature :hasTemperatureSensor\_Observe\_SolarPanelComposition\_observed :part . \newline
\hspace*{1.0em}?obs\_2\_temperature :haspartParticipant\_assembly :Panel\_PartOf\_SolarPanel . \newline
\hspace*{1.0em}?obs\_2\_temperature :hasPanel\_PartOf\_SolarPanelComposition\_part :Panel . \newline
\hspace*{1.0em}?obs\_3\_angle a :Observation . \newline
\hspace*{1.0em}?obs\_3\_angle :hasObservable :Angle . \newline
\hspace*{1.0em}?obs\_3\_angle :hasSatellite :Satellite . \newline
\hspace*{1.0em}?obs\_3\_angle :hasSatelliteParticipant\_assembly :SunSenor\_PartOf\_Satellite . \newline
\hspace*{1.0em}?obs\_3\_angle :hasSunSenor\_PartOf\_SatelliteComposition\_part :SunSensor . \newline
\hspace*{1.0em}?obs\_3\_angle :hasSunSensorParticipant\_observer :SunSensor\_Observe\_IncidenceAngle . \newline
\hspace*{1.0em}?obs\_3\_angle :hasSunSensor\_Observe\_IncidenceAngleComposition\_observed :SolarPanel\_IncidenceAngle\_Sun . \newline
\hspace*{1.0em}?obs\_4\_operationalState a :Observation . \newline
\hspace*{1.0em}?obs\_4\_operationalState :hasObservable :OperationalState . \newline
\hspace*{1.0em}?obs\_4\_operationalState :hasSatellite :Satellite . \newline
\hspace*{1.0em}?obs\_4\_operationalState :hasSatelliteParticipant\_assembly :SolarPanel\_PartOf\_Satellite . \newline
\hspace*{1.0em}?obs\_4\_operationalState :hasSolarPanel\_PartOf\_SatelliteComposition\_part :SolarPanel . \newline
\}} \\
\hline
\end{tabular}
    \caption{Example UDDL Queries and Generated SPARQL.}
    \label{tab:query_examples}
\end{table*}

\section{Analysis} \label{sec:analysis}

TODO

\section{Discussion} \label{sec:discussion}

Can we add constraints on constructions of ontologies that force conformance with UDDL data models?


\section{Conclusion} \label{sec:conclusion}

TODO


\section{Acknowledgments} \label{sec:acknowledgements}

This work was conducted using Prot\'eg\'e.

\newpage
\section{References}
\printbibliography[heading=none]

% ---------- Biography Format ----------
\newpage
\phantomsection
\makeatletter
\renewcommand{\authorbioentry}[3]{%
  \noindent\begin{tabular}{@{}m{0.5in} M{\dimexpr\columnwidth-0.5in\relax}@{}}
    \authorpic{#1} &
    {\headingfont\bfseries\raggedright\fontsize{12pt}{14pt}\selectfont #2}\par #3
  \end{tabular}\par\medskip
}
\makeatother
% ---------- Biography ----------
% ---- Note : Begin the Biography section on a new page

\section*{Biography}
\authorbioentry{template-images/author1_pic.jpg}{Author Name}{Provide a short biography of the author. Provide a short biography of the author.}
\authorbioentry{template-images/author2_pic.jpg}{Second Author}{Provide a short biography of the second author.}
\authorbioentry{template-images/author3_pic.jpg}{Third Author}{Provide a short biography of the third author.}


\newpage

\section{Appendix: Satellite System ICD} \label{sec:Appendix}

\subsection{Passive Thermal Control: Venetian-blind Louvers} \label{section:venetian_blind_louvers}

Louvers are mechanical components that passively aid the regulation of thermal convection between the satellite's internal components (payload, comm system, etc.) and the outer space vacuum. Louvers are usually employed on satellites because of their low mass, and their operation relies solely on temperature variation rather than generated power, leaving margin in the satellite power and mass budget.

\subsection{Location on Satellite and Systems Interactions} \label{section:louvers_location}

Louvers are mostly part of an external radiator assembly (being located on top of it), which is placed on the satellite flat panels or on deployable panels. Louvers could also be placed internally between surfaces to move the heat from the heat sources to radiators and ultimately away from the satellite. Or in general terms, they are located on top of a heat source while facing a cold sink (space vacuum).

\begin{table}[H]
    \centering
    \renewcommand{\arraystretch}{1.5}
    \begin{tabularx}{\columnwidth}{@{}P{0.45\columnwidth}Y@{}}
        \rowcolor{tableheader}
        \multicolumn{1}{c}{\headingfont\bfseries Interacting Systems} & \multicolumn{1}{c}{\headingfont\bfseries Primary Assignment} \\
        \addlinespace[4pt]
        Thermal Control & Thermal regulation. \\
        \addlinespace[4pt]
        Onboard Computer (OBC) & Temperature monitoring.
    \end{tabularx}
    \caption{Venetian-blind Louvers Interacting Systems}
    \label{tab:Venetian_blinds}
\end{table}


\subsection{Louver Sub-system Multiplicity} \label{section:louvers_multiplicity}

The satellite may include one or more louvers and each one of them operates individually and send its own telemetry messages.

\subsection{Louver Mechanical Assembly Components and Interfaces} \label{section:louvers_components}

Louvers are made of three main components
\begin{itemize}
    \item The structural frame is mounted to a heat source within the satellite (internal or external).
    \item Louver blades are the elements that open and close, regulating heat convection and emissivity levels of the heat source of interest.
    \item Bimetallic Springs (made of two different aluminium alloys) are temperature-sensitive springs that connect the structural frame (in touch with the heat source) with the blades. They are bimetal to ensure that their bending will effectively open the blades at an angle. The spring materials have a set point temperature between around \SI{-20}{\celsius} and \SI{50}{\celsius} when the blades start to open. Then the blades go from fully closed to fully open within a \SI{10}{\celsius} to \SI{20}{\celsius} control band~\parencite{SierraSpace_Passive_Thermal_Louvers_2025}.
\end{itemize}

\begin{table}[H]
    \centering
    \renewcommand{\arraystretch}{1.5}
    \begin{tabularx}{\columnwidth}{@{}P{0.45\columnwidth}Y@{}}
        \rowcolor{tableheader}
        \multicolumn{1}{c}{\headingfont\bfseries Interface} & \multicolumn{1}{c}{\headingfont\bfseries Applicable to subsystem} \\
        \addlinespace[4pt]
        Telemetry & \checkmark \\
        \addlinespace[4pt]
        Command & $\times$ \\
        \addlinespace[4pt]
        Thermal & \checkmark \\
        \addlinespace[4pt]
        Mechanical & \checkmark \\
        \addlinespace[4pt]
        Power & $\times$
    \end{tabularx}
    \caption{Venetian-blind Louvers Interface Overview}
    \label{tab:louvers_interfaces}
\end{table}

\subsection{Mechanical Operation and Telemetry Messages} \label{section:louvers_operation}

During periods of higher operational needs (e.g. increase of data transmission between the communication subsystem and the ground station) the source of interest heats up, consequently increasing the temperature within the louvers' structural frame. The bimetallic springs, being installed on the frame, will consequently change its state. As the springs receive heat, the metals that compose them will expand at different rates, which allows the blades to mechanically open. By rotating, they expose the heat source surface (high emissivity) to the cold sink so that more heat can be irradiated out into the space vacuum, without overheating the source and compromising critical operations.

When the heat source and the structural frame cool down, consequently, the springs will compress to close the blades, reducing heat dissipation (low-emissivity) and keeping the source warmer during low-demand periods.


\begin{table*}[t]
    \centering 
    \small
    \renewcommand{\arraystretch}{1.5}
    \begin{tabularx}{\textwidth}{@{}A{0.18\textwidth}A{0.16\textwidth}>{\raggedright\arraybackslash}XA{0.15\textwidth}A{0.15\textwidth}A{0.08\textwidth}@{}}
        \rowcolor{tableheader}
        \headingfont\bfseries Telemetry Message & \headingfont\bfseries Trigger & \headingfont\bfseries Notes & \headingfont\bfseries Sender & \headingfont\bfseries Receiver & \headingfont\bfseries Multiplicity \\
        \addlinespace[4pt]
        Louvers\_status\_info & Periodic \& out-of-range reach alert. & Temperature rise or drop outside the set range. Set range: \SI{-20}{\celsius} to \SI{50}{\celsius} & Louver assembly ID & Thermal Control \& OBC & 1 \\
        \addlinespace[4pt]
        Louvers\_health\_info & Periodic \& Mechanical fault alert. &  & Louver assembly ID & OBC & 1
    \end{tabularx}
    \caption{Venetian-blind Louvers Telemetry Message Overview}
    \label{tab:louvers_tm_overview}
\end{table*}


\begin{table*}[t]
    \centering
    \small
    \renewcommand{\arraystretch}{1.5}
    \begin{tabularx}{\textwidth}{@{}A{0.22\textwidth}>{\raggedright\arraybackslash}XA{0.10\textwidth}A{0.18\textwidth}A{0.15\textwidth}@{}}
        \rowcolor{tableheader}
        \headingfont\bfseries ID & \headingfont\bfseries Description & \headingfont\bfseries Data type & \headingfont\bfseries Units & \headingfont\bfseries Update Periodicity \\
        \addlinespace[4pt]
        Louver\_assembly\_id & Distinctive louver identifier. & Int &  & 1 Hz \\
        \addlinespace[4pt]
        Frame\_temperature & Measured through a temperature sensor mounted on the louver's structural frame. & Float & \SI{}{\celsius} & Every 1 second \\
        \addlinespace[4pt]
        Blade\_angle & Measured through a position sensor. & Float & \SI{}{\degree} & Every 1 second \\
        \addlinespace[4pt]
        Emissivity\_state & Derived parameter from the position sensor. & Enum & Low, Medium, High & Every 1 second
    \end{tabularx}
    \caption{Venetian-blind Louvers Status Information}
    \label{tab:louvers_status}
\end{table*}


\begin{table*}[t]
    \centering
    \small
    \renewcommand{\arraystretch}{1.5}
    \begin{tabularx}{\textwidth}{@{}A{0.22\textwidth}>{\raggedright\arraybackslash}XA{0.10\textwidth}A{0.18\textwidth}A{0.15\textwidth}@{}}
        \rowcolor{tableheader}
        \headingfont\bfseries ID & \headingfont\bfseries Description & \headingfont\bfseries Data type & \headingfont\bfseries Units & \headingfont\bfseries Update Periodicity \\
        \addlinespace[4pt]
        Louver\_assembly\_id & Distinctive louver identifier. & Int &  & 1 Hz \\
        \addlinespace[4pt]
        Health\_check & Overall structural integrity. & Enum & Nominal, Attention, Fault & Every 2 min \\
        \addlinespace[4pt]
        Blade\_mech\_compliance & The blades' smooth movement during operation. & Enum & Nominal, Attention, Fault & Every 2 min
    \end{tabularx}
    \caption{Venetian-blind Louvers Health Information}
    \label{tab:louvers_health}
\end{table*}

\subsection{Power Generation System: Solar Panels} \label{section:solar_panels}

A Solar Panel (also known as a photovoltaic cell or solar cell) is a flat surface device that directly transforms solar radiation into electrical power. Earth-orbiting satellites from LEO (Low Earth Orbit) to GEO (Geosynchronous Orbit) are the best candidates for solar panels as their power source.  Solar panels are used to convert sunlight into electrical power that the satellite systems and subsystems use to support their operations during the mission lifetime. The solar energy converted by the panels can also be stored in onboard batteries that will provide electrical energy to the satellite systems while in the Earth's shadow.

\subsection{Location on Satellite and Systems Interaction} \label{section:solar_panels_location}

Solar panels are placed on the satellite exterior, flat panels or on deployable paddles that unfold once the satellite reaches its intended orbit. Their positioning is meant to maximize their sun exposure for uninterrupted power generation before passing to the Earth's shadow.

\begin{table}[H]
    \centering
    \renewcommand{\arraystretch}{1.5}
    \begin{tabularx}{\columnwidth}{@{}P{0.45\columnwidth}Y@{}}
        \rowcolor{tableheader}
        \multicolumn{1}{c}{\headingfont\bfseries Interacting Systems} & \multicolumn{1}{c}{\headingfont\bfseries Primary assignment} \\
        \addlinespace[4pt]
        ADCS computer & Controls attitude to improve the Sun incidence on panels. \\
        \addlinespace[4pt]
        Onboard Computer (OBC) & Collect telemetry data, monitor power budgeting, and health of the panels. \\
        \addlinespace[4pt]
        Power Conditioning and Distribution Unit (PCDU) & Collect electricity generated by solar panels and distribute it to the satellite systems and the battery. \\
        \addlinespace[4pt]
        Mechanical Structure & Provides a stable structural support and alignment with the satellite reference frame.
    \end{tabularx}
    \caption{Solar Panels Interacting Systems}
    \label{tab:solar_interactions}
\end{table}

\subsection{Solar Panel Sub-system Multiplicity} \label{section:solar_multiplicity}

Multiple. There are usually more than one solar panel (fixed on a satellite panel or deployable) and each one of them operates individually and sends its own telemetry messages.

\subsection{Solar Panels Assembly Components and Interfaces} \label{section:solar_components}

\paragraph{Mechanical Components}
\begin{itemize}
    \item The deployable array/structural frame is a wing-like structure that connects multiple solar panels. During launch the deployable array are stowed and deploy once in orbit. It maintains panel pointing while surviving launch loads and on-orbit Sun incidence.
    \item The substrate is the panel backbone, is made of aluminium honeycomb core with composite or aluminum sheets, onto which the photovoltaic assembly is laced. It provides mechanical support, electrical insulation layers, and a flat surface to mount solar cells and interconnects.
    \item Coverglasses are thin radiation-resistant glass plates laced on top of each cell to protect against UV, ionizing radiation, and micrometeoroid/particle impacts. They also include optical coatings to reduce reflection.
    \item Interconnects are metallic conductors (ribbons or welds) that join cells into strings and connect strings to bus bars. Their geometry and material minimize resistive losses while allowing thermal expansion and mechanical flexibility.
    \item Encapsulant is the adhesive/potting material between cells, coverglass, and substrate that bonds and seals them from the space environment. It helps the stack tolerate wide temperature swings, outgassing limits, and radiation without embrittling or losing adhesion.
\end{itemize}
\paragraph{Electrical Components}
\begin{itemize}
    \item Solar cells are the photovoltaic devices that convert incident sunlight into DC electrical power. Cells are arranged into series/parallel to achieve the required array voltage and current for powering the satellite bus.
    \item Blocking diodes are placed in series with strings to prevent reverse current flow when parts of the array are shaded or damaged.
    \item The junction box is the interface node where strings come together, housing terminations, blocking/bypass diodes, and connectors to the satellite power provider~\parencite{barrett1971spacecraft}.
\end{itemize}

\begin{table}[H]
    \centering
    \renewcommand{\arraystretch}{1.5}
    \begin{tabularx}{\columnwidth}{@{}P{0.45\columnwidth}Y@{}}
        \rowcolor{tableheader}
        \multicolumn{1}{c}{\headingfont\bfseries Interface} & \multicolumn{1}{c}{\headingfont\bfseries Applicable to subsystem} \\
        \addlinespace[4pt]
        Telemetry & \checkmark \\
        \addlinespace[4pt]
        Command & \checkmark  \\
        \addlinespace[4pt]
        Thermal & \checkmark  \\
        \addlinespace[4pt]
        Mechanical & \checkmark  \\
        \addlinespace[4pt]
        Power/Electrical & \checkmark
    \end{tabularx}
    \caption{Solar Panels Interface Overview}
    \label{tab:solar_interfaces}
\end{table}

\subsection{Operation and Telemetry Messages}

Sun radiation reaches at a determined angle the solar cells, where photons excite the electron in the semiconductor and generates a \SI{2}{\volt} voltage per cell.  The electrical path stability is kept by the connection between Cells and the coverglass, with the substrate. The junction box collects the transformed solar energy and the panel strings terminate and combine to form the array output lines that feed the satellite power system. From the junction box, the generated power goes to the power management and distribution unit, which regulates the voltage, charges the batteries, and switches power to the different systems such as communications, payload, and attitude control subsystems. During periods of sunlight, solar power runs the satellite and charges the batteries; while during period of shadows, the batteries are the only supplier of power to the bus~\parencite{reddy2008planning}. 


\begin{table*}[t]
    \centering
    \small
    \renewcommand{\arraystretch}{1.5}
    \begin{tabularx}{\textwidth}{@{}A{0.22\textwidth}A{0.28\textwidth}>{\raggedright\arraybackslash}XA{0.14\textwidth}A{0.12\textwidth}A{0.10\textwidth}@{}}
        \rowcolor{tableheader}
        \headingfont\bfseries Telemetry Message & \headingfont\bfseries Trigger & \headingfont\bfseries Notes & \headingfont\bfseries Sender & \headingfont\bfseries Receiver & \headingfont\bfseries Multiplicity \\
        \addlinespace[4pt]
        SolarPanel\_config\_state & Periodic \& mode change occurrence. &  & Solar Panel & OBC & 1 \\
        \addlinespace[4pt]
        SolarPanel\_health\_info & Periodic \& Mechanical/Electrical fault alert. &  & Solar Panel & OBC & 1
    \end{tabularx}
    \caption{Solar Panel Telemetry Message Overview}
    \label{tab:solar_panel_tm_overview}
\end{table*}

\begin{table*}[t]
    \centering
    \small
    \renewcommand{\arraystretch}{1.5}
    \begin{tabularx}{\textwidth}{@{}A{0.22\textwidth}>{\raggedright\arraybackslash}XA{0.08\textwidth}A{0.08\textwidth}A{0.25\textwidth}@{}}
        \rowcolor{tableheader}
        \headingfont\bfseries ID & \headingfont\bfseries Description & \headingfont\bfseries Data type & \headingfont\bfseries Units & \headingfont\bfseries Update Periodicity \\
        \addlinespace[4pt]
        SolarPanel\_id & Distinctive solar panel identifier. & Int &  & 1 Hz \\
        \addlinespace[4pt]
        SolarPanel\_deploy & Indicates deployment event. & Enum & Nominal, Attention, Fault & Periodic (at deployment), 1 per panel \\
        \addlinespace[4pt]
        SolarPanel\_temp & Measured panel temperature from integrated sensors. & Float & \SI{}{\celsius} & Periodic, 1 Hz per sensor \\
        \addlinespace[4pt]
        SolarPanel\_incid & Measures the angle of incidence between the sun and the panel. & Enum & \SI{}{\degree} & Periodic, 1 per panel \\
        \addlinespace[4pt]
        SolarPanel\_ops\_mode & Indicates how solar panels process power. & Enum & Off, safe, eclipse, fixed V, max power & Prompt by mode change \& 1 Hz
    \end{tabularx}
    \caption{Solar Panel Configuration State}
    \label{tab:solar_panel_configuration}
\end{table*}

\begin{table*}[t]
    \centering
    \small
    \renewcommand{\arraystretch}{1.5}
    \begin{tabularx}{\textwidth}{@{}A{0.22\textwidth}>{\raggedright\arraybackslash}XA{0.10\textwidth}A{0.18\textwidth}A{0.12\textwidth}@{}}
        \rowcolor{tableheader}
        \headingfont\bfseries ID & \headingfont\bfseries Description & \headingfont\bfseries Data type & \headingfont\bfseries Units & \headingfont\bfseries Update Periodicity \\
        \addlinespace[4pt]
        SolarPanel\_id & Distinctive solar panel identifier. & Enum &  &  \\
        \addlinespace[4pt]
        SolarPanel\_temperature & Solar array electronics internal temperature. & Float & \SI{}{\celsius} & 10 Hz \\
        \addlinespace[4pt]
        SolarPanel\_voltage\_in & Solar array voltage input. & Float & \SI{}{\volt} & 10 Hz \\
        \addlinespace[4pt]
        SolarPanel\_current\_out & Solar panel output current. & Float & \SI{}{\milli\ampere} & 10 Hz \\
        \addlinespace[4pt]
        SolarPanel\_tot\_power & Total power generated by solar array. & Float & \SI{}{\watt} &  \\
        \addlinespace[4pt]
        Health\_check & Internal health monitoring. & Enum & Nominal, Attention, Fault & 10 Hz \\
        \addlinespace[4pt]
        SolarPanel\_degradation & Power degradation. & Float & Nominal, Attention, Fault & 0.1 Hz
    \end{tabularx}
    \caption{Solar Panel Health Information}
    \label{tab:solar_panel_health}
\end{table*}

 
\end{multicols*}
\end{document}
