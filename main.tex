\documentclass[12pt]{article}

% Essential packages
\usepackage[utf8]{inputenc}
\usepackage[T1]{fontenc}
\usepackage{graphicx}  % For including PNG figures
\usepackage{float}     % For better figure placement
\usepackage{amsmath}   % For mathematical equations
\usepackage{amsfonts}
\usepackage{amssymb}
\usepackage{hyperref}  % For hyperlinks
\usepackage{url}       % For URLs in bibliography
\usepackage{authblk}   % For author affiliations
\usepackage{longtable} % For multi-page tables

\DeclareMathOperator*{\argmin}{\arg\!\min}
\DeclareMathOperator*{\minimize}{\text{minimize}}

% Bibliography
\usepackage[style=numeric,backend=biber]{biblatex}
\addbibresource{ref.bib}

% Page setup
\usepackage[margin=1in]{geometry}
\usepackage{setspace}
\usepackage{parskip}
\onehalfspacing

% Title information
\title{Aligning the Universal Domain Description Language (UDDL) with Semantic Web Technologies}
\author[1]{Nicholas McKibben}
\author[2]{Joe Gregory}
\author[3]{Alejandro Salado}
\affil[1,2,3]{Department of Systems and Industrial Engineering, University of Arizona}
\affil[1]{Skayl}
\date{\today}

\begin{document}

\maketitle

\begin{abstract}
The engineering of complex, software-intensive systems faces a pervasive challenge: data ambiguity~\cite{INCOSE2035, madni2014system}.  As disparate systems ranging from avionics to Industrial Internet of Things (IIoT) attempt to exchange information, the semantic context of that data is often not communicated clearly and effectively, leading to additional unplanned integration effort and interoperability gaps~\cite{noura2019interoperability, mzili2025interoperability, fusco2020approach}.  The Universal Domain Description Language (UDDL) addresses this by providing a platform-independent data modeling language rooted in the separation of concerns between Conceptual, Logical, and Platform data models~\cite{OpenGroup2023UDDL}.  However, UDDL constructs are not well-studied in systems and interoperability literature while the Web Ontology Language (OWL) \cite{W3C_OWL2} and the Resource Description Framework (RDF) \cite{W3C_RDF11} (the de facto standard for semantic knowledge representation on the web) are actively researched and have rich tooling ecosystems.  We recognize UDDL use-cases as specializations of Sematic Web Technologies (SWTs) use-cases and that by bridging technology concepts both communities stand to gain insights in creating and utilizing data semantics. 

We therefore propose a mapping between UDDL and SWTs, specifically addressing the UDDL conceptual model elements capturing data semantics.  These mappings provide a blueprint for a UDDL to SWTs bridge that retains necessary semantic rigor while leveraging the inferential power of the Semantic Web ecosystem.
\end{abstract}

\section{Introduction} \label{section:introduction}

In systems engineering, rigorous definition of data and interfaces is not merely a documentation exercise, but a fundamental requirement for the composition, maintenance, and future reuse of complex systems~\cite{fusco2020approach, madni2014system, Friedenthal2014, INCOSE2035}.  In traditional systems engineering, data dictionaries and interface control documents (ICDs) have served as the primary means of semantic capture~\cite{Friedenthal2014, price2013semantic}.  However, these artifacts are often static in nature and reliant on human interpretation~\cite{Friedenthal2014}.  The UDDL Technical Specification recognizes that ``[adequately] describing domain data is a common problem in computer-based systems.  Typical approaches involve creating documentation to accompany artifacts \ldots in an attempt to capture the semantics of the data exchanged''~\cite{OpenGroup2023UDDL}.

The failure of traditional, document-based approaches to capture data semantics stems from their inability to formally and precisely bind meaning to structure, i.e., attach semantics to syntax~\cite{price2013semantic, Friedenthal2014}.  For example, a variable labeled \texttt{alt} in a ICD may imply ``altitude'', but it may fail to specify a reference frame (e.g., ellipsoidal vs. geoidal), the unit of measure (e.g., meters vs. feet), or the domain (e.g., air operations vs. ground).  UDDL was developed expressly to fill this documentation gap, its objective being to ``formally document the meaning of data with the goal of eliminating ambiguity''~\cite{OpenGroup2023UDDL}.  It achieves this by providing the means to create conceptual building blocks that are then used to construct formal definitions of the meaning of data flowing through interfaces.

Parallel to the development of UDDL, the Semantic Web has evolved a robust stack of technologies designed for machine-readability and logical inference~\cite{berners2023semantic}.  At the core of this stack are RDF (Resource Description Framework) and OWL (Web Ontology Language).  RDF provides a graph-based data model where knowledge is encoded as tuples (Subject, Predicate, Object), offering high flexibility and linkability~\cite{W3C_RDF11}.  OWL extends RDF with formal logical semantics (Description Logics), enabling systems to define complex relationships, cardinality constraints, and classifications~\cite{W3C_OWL2}.

Despite development and growing industry use of UDDL and its predecessor data modeling languages since 2013~\cite{OpenGroup2013FACE}, to our knowledge, there is no existing literature showing interoperability, traceability, or mappings between UDDL data models and any Semantic Web Technology.  This paper serves as an initial attempt at formally bridging the technologies and approaches of two disparate communities of research and industry.

The remainder of the Introduction will provide brief overviews of UDDL  and Semantic Web Technologies to serve as context for the mappings described in the Theory section.  Section \ref{section:introduction-uddl-overview} introduces the UDDL meta-model and grammar that specify its data modeling language.  Section \ref{section:introduction-swts-overview} describes the SWTs that will be useful in mapping UDDL concepts.

\subsection{UDDL} \label{section:introduction-uddl-overview}

\subsubsection{Meta-Model Overview}

The UDDL utilizes a multi-level modeling approach summarized in Table \ref{tab:uddl_levels}~\cite{OpenGroup2023UDDL}:

\begin{table}[H]
    \centering
    \begin{tabular}{|p{0.25\linewidth}|p{0.35\linewidth}|p{0.3\linewidth}|}
        \hline
        \textbf{Model Level} & \textbf{Description} & \textbf{Semantic Focus} \\
        \hline
        Conceptual Data Model (CDM) & Defines the complete semantics of all Entities and Associations. & Concepts and relationships.  Entities and Associations are typed with Observables (traits which can be observed but not further characterized). \\
        \hline
        Logical Data Model (LDM) & Refines Entities/Associations with Measurements. & Quantification.  Further characterizes Observables by adding frames of reference, units, and value types (e.g., Integer, Real). \\
        \hline
        Platform Data Model (PDM) & Refines elements with Platform Data Types. & Implementation.  Additional characterization of logical measurements to include bit-level precision and language bindings in terms of IDL. \\
        \hline
    \end{tabular}
    \caption{UDDL Multi-level Modeling Approach.}
    \label{tab:uddl_levels}
\end{table}

This paper will focus primarily on the Conceptual Data Model (CDM), as the meaning of the data (i.e., the semantics) are completely described by conceptual elements and their relationships.  We will draw upon lower-level LDM and PDM concepts to justify design decisions to ensure the mappings are compatible with lower-level refinements.  Therefore, we will review only the conceptual meta-model elements and their relationships.

Table \ref{tab:cdm_elements} informally summarizes the construction of relevant conceptual meta-model elements.  All elements are assumed to have a globally unique name as well as an optional textual description field.  A Conceptual Data Model (CDM) is a top-level package that collects and organizes all conceptual model elements.  CDMs can be nested to create hierarchies or other organizational structures and multiple CDMs may be included in a single root-level Data Model.  A Domain collects a set of Basis Entities, relating to well understood concepts by practitioners within a particular domain~\cite{OpenGroup2023UDDL}.  Basis Entities represent unique concepts that are not expressible simply through model structure, e.g., the concepts of ``Minimum'' and ``Maximum''~\cite{sdm_v3_1_15}.  Observables are things that can be observed or measured, e.g., a Position, Orientation, Identifier, and are used to ``type'' Entities.  Notice the examples for Basis Entity and Observable in table \ref{tab:cdm_elements} -- a ``Position'' is something that can be measured and is therefore an Observables while ``Place'' is a concept and not meaningfully measurable and is therefore a Basis Entity (i.e., you can not measure ``Place'' directly, but you could measure the Position of a Place, or the Extent of a Place, Position and Extent being Observables).  Entities represent a domain concept in terms of its Observables and other composed conceptual Entities.  Associations are themselves Entities but which also include two or more Participants that can associate other Entities as well as Observables.  The Participants in the Association are specified through Participant Paths which are a sequence of nodes that specify an arbitrary path through the network of related Entities and Associations and following traversal rules, namely that Compositions can be traversed in the direction of the composition and Participants of Associations can be traversed bidirectionally.  Participant Paths are not restricted otherwise and may contain cycles.

\begin{longtable}{|p{0.25\linewidth}|p{0.45\linewidth}|p{0.2\linewidth}|}
    \caption{Core UDDL Conceptual Meta-Model Elements.} \label{tab:cdm_elements} \\
    \hline
    \textbf{Element Name} & \textbf{Construction} & \textbf{Example} \\
    \hline
    \endfirsthead
    
    \multicolumn{3}{c}%
    {{\bfseries \tablename\ \thetable{} -- continued from previous page}} \\
    \hline
    \textbf{Element Name} & \textbf{Construction} & \textbf{Example} \\
    \hline
    \endhead
    
    \hline \multicolumn{3}{|r|}{{Continued on next page}} \\ \hline
    \endfoot
    
    \hline
    \endlastfoot

    Conceptual Data Model & Package conceptual model elements in a single collection. & \texttt{Observables} (a collection of all Observable data model elements). \\
    \hline
    Domain & A collection of Basis Entities. & \texttt{Sensing}; includes the \texttt{Object} Basis Entity~\cite{sdm_v3_1_15}. \\
    \hline
    Basis Entity & Unique axiomatic domain concept. & \texttt{Place}~\cite{sdm_v3_1_15} \\
    \hline
    Observable & Unique trait which can be observed but not further characterized. & \texttt{Position}~\cite{sdm_v3_1_15} \\
    \hline
    Entity & A domain concept characterized by a set of composed Observables and/or other Entities.  Can Specialize another Entity or Domain Entity. & \texttt{Person}; composes \texttt{Identifier} and \texttt{Position} Observables. \\
    \hline
    Association & A domain concept that is expressible only through the relation of two or more Participants (Entities and/or Observables).  Associations may also compose a set of Entities and/or Observables and Specialize another Entity or Domain Entity. & \texttt{Ownership}; composes \texttt{Duration} and associates \texttt{Person}, \texttt{Car} Entities through \texttt{owner}, \texttt{asset} Participant roles. \\
\end{longtable}

\begin{table}[H]
    \centering
    \caption{UDDL Conceptual Relationships.} \label{tab:cdm_relationships}
    \begin{tabular}{|p{0.25\linewidth}|p{0.45\linewidth}|p{0.2\linewidth}|}
        \hline
        \textbf{Relation} & \textbf{Construction} & \textbf{Example} \\
        \hline
        Composition & Aggregation of conceptual elements (Entities/Observables) into a parent Entity.  Includes multiplicity specifications (upper and lower bounds).  Can be travered by Participant Paths in the direction of the composition, i.e., from parent to child. & \texttt{Car} composes at most $ 4 $ \texttt{Wheels}. \\
        \hline
        Participant & Role-based link in an Association pointing to an Entity/Observable via a path.  Includes source/target multiplicities (upper and lower bounds).  Can be traversed bidirectionally by Participant Paths. & In the Association \texttt{Teaching}, the participant Entity \texttt{Person} has a single role of \texttt{instructor} and a second participant \texttt{Person} has role of \texttt{student}, potentially many. \\
        \hline
        Generalization & ``is-a'' relationship where a specialized Entity declares it is a specialization of another Entity.  Cannot be traversed by Participant Paths. & \texttt{FixedWingAircraft} specializes \texttt{Aircraft}. \\
        \hline
        Realization & Refinement relationship binding abstract concepts to more specific definitions across model levels (CDM $\to$ LDM).  Cannot be traversed by Participant Paths (no meaning by doing this). & \texttt{AltitudeMeters} (LDM) realizes \texttt{Altitude} (CDM). \\
        \hline
    \end{tabular}
\end{table}

Table \ref{tab:cdm_relationships} summarizes how relationships between core UDDL Conceptual Meta-Model Elements are constructed.  Composition relationships include an optional multiplicity specifications to indicate the minimum and maximum number of elements that can be composed ($ -1 $ indicates unbounded).  If not specified, the default is one.  Similarly, Participants of Associations include optional source and target multiplicities that are applied when calculating the multiplicities along the Participant Paths.  Participant multiplicities are assumed to be unbounded if not specified.

% generated figure from entity-association-explainer.mmd
\begin{figure}[H]
    \centering
    \includegraphics[width=\textwidth]{entity-association-explainer.png}
    \caption{Example of some features of UDDL Entity construction.  Gray boxes are composed Entities and Observables, e.g., \texttt{EntityX} composes \texttt{ObservableA} and \texttt{ObservableB}.  Solid, dotted, and dashed lines represent Participant Paths.  The Participant Path associated with \texttt{participant3} traverses two nodes, through the composed \texttt{EntityZ} of \texttt{EntityY} and terminating on the \texttt{ObservableD} of \texttt{EntityZ}. No multiplicities are specified for compositions and participant paths and should therefore default multiplicities should be assumed. }
    \label{fig:entity-association}
\end{figure}

Figure \ref{fig:entity-association} provides a summary of some of the feaures of Entity and Associations construction.  Entities \texttt{EntityX}, \texttt{EntityY}, and \texttt{EntityZ} compose Observables and other Entities (grey boxes).  The single \texttt{Association} associates Entities and Observables through three different Participant Paths represented by solid, dotted, and dashed lines.  \texttt{participant1} simply associates the entire \texttt{EntityX}.  \texttt{participant2} specifically associates the \texttt{ObservableC} of \texttt{EntityY}.  \texttt{participant3} associates the \texttt{ObservableD} of \texttt{EntityZ} through the composed \texttt{EntityZ} within \texttt{EntityY}, demonstrating a Participant Path with two nodes.

\subsubsection{Query Construction}

The UDDL Query grammar and associated construction rules defines a SQL-like language for specification of data semantics~\cite{OpenGroup2023UDDL}.  Query specifications can be constructed at all three levels of the UDDL meta-model.  The only differences in expression of query specifications at the various levels are the meta-types of Entities and Associations (CDM, LDM, PDM; CDM used for conceptual query specifications, LDM for logical query specifications, etc.) and qualifiers that may be used (e.g., in a \texttt{WHERE} clause).  \texttt{WHERE} clauses and other qualifiers specify data filtering criteria and side-effects on generated syntax when combined with a FACE Template specification~\cite{OpenGroup2017FACE} but do not contribute to the semantics of the query.  We therefore need not consider mapping of \texttt{WHERE} clauses and other qualifiers, including sub-query specifications.  Entities and Associations will map cleanly across realizations between levels of the UDDL meta-model up to down-selection.  Considering this along with the fact that Generalization relationships must also be realized across refinement levels (up to down-selection), we will without loss of generality consider only conceptual query specifications and \texttt{SELECT} and \texttt{JOIN} query language elements.

\subsection{Semantic Web Technologies} \label{section:introduction-swts-overview}

The Semantic Web technology stack provides a layered architecture for machine-readable knowledge representation.  Table \ref{tab:swt_overview} summarizes the core technologies relevant to this mapping.

\begin{table}[H]
    \centering
    \begin{tabular}{|p{0.15\linewidth}|p{0.75\linewidth}|}
        \hline
        \textbf{Technology} & \textbf{Description} \\
        \hline
        RDF & The Resource Description Framework provides a graph-based data model where knowledge is represented as triples (Subject, Predicate, Object), serving as the foundation for data interchange. \\
        \hline
        OWL & The Web Ontology Language extends RDF with formal Description Logic semantics, enabling the definition of complex class hierarchies, restrictions, and inferential rules. \\
        \hline
        SPARQL & The standard query language for RDF, allowing for the retrieval and manipulation of data stored in RDF format. \\
        \hline
        SHACL & The Shapes Constraint Language provides a mechanism to validate RDF graphs against a set of conditions (shapes), ensuring structural conformance. \\
        \hline
    \end{tabular}
    \caption{Overview of Semantic Web Technologies.}
    \label{tab:swt_overview}
\end{table}

TODO: ask Joe to flesh this out a bit more.

\section{Theory}

In this section we present mappings between UDDL and SWTs along with supporting justifications.  In general, we will rely on RDF and OWL for core UDDL element and relationship mappings.  For UDDL Queries we will leverage additional SWTs including SPARQL and SHACL.

\subsection{Closed-World and Open-World Assumptions} \label{section:theory-cwa-owa}

UDDL is used to construct a domain-specific data models (DSDMs), functioning as a descriptive language to talk about all the data flowing through system interfaces.  Any concept not able to be captured using this data model definitionally cannot be understood by any interfaced component in the system if the interfaces are fully characterized using a common data model.  We therefore must have a closed-world assumption (CWA) when using UDDL, i.e., the absence of knowledge implies falsity, or more simply, that something not specified is assumed not to exist.  If some modeled object \texttt{Car} does not have a property \texttt{Wings}, then a \texttt{Car} categorically cannot have wings.  Even if a car with wings did interact with the system, the system must act as if it did not have wings as no interface is able to consider cars of winged type.

In SWTs, we make an open-world assumption (OWA), i.e., the absence of knowledge does not imply falsity.  This is central to the idea of the semantic web, as we must assume we can find information elsewhere.  If an ontology does not state that a \texttt{Car} cannot have \texttt{Wings}, an inference engine must assume a car could potentially have wings.

Because of the CWA in UDDL and OWA in SWTs, in order to represent UDDL concepts using SWTs we must add ``Closure Axioms'' (restrictions) to simulate the structural rigor of UDDL within the open world.

% This will also complicate our mapping from SWTs to UDDL and forces us to admit that not all ontological structures are mappable back to UDDL.

\subsection{Containers}

The conceptual:DataModel UDDL element acts as a wrapper for sub-models.  In OWL, this is best captured as a ``Root'' ontology that simply imports the constituent CDM, LDM, and PDM ontologies to provide a unified view.

The CDM defines entities and observables independent of any specific protocol or storage representation.  This can be captured as a Domain Ontology that defines the Classes and Properties (concepts and relationships) of the universe of discourse.

The LDM refines the CDM by adding ``terms of Measurement Systems \ldots and Units''.  This maps to an ontology that imports the Domain Ontology (CDM) and specializes it with measurement constraints (e.g., using QUDT or OM units) and precision rules (using SHACL shapes for constraints).

The PDM defines ``specific representation details such as data type and precision'' (e.g., float vs. double).  This maps to an implementation-specific ontology that binds logical concepts to XSD datatypes.  No specific serialization schema is required at this level.

\subsection{Generalization}

UDDL Generalization can be represented as a subclassing relationship in RDF, i.e., \texttt{rdfs:subClassOf} represents inclusion logic (an instance of the subclass is necessarily an instance of the superclass).  Note that in Data Modeling and Logic, Generalization implies Set Inclusion.  For example, if \texttt{Jet} specializes \texttt{Aircraft}, then every instance of \texttt{Jet} is also considered an instance of \texttt{Aircraft}.  \texttt{rdfs:subClassOf} does not imply polymorphism.

\subsection{Realization}

In OWL, this refinement relationship can be modeled using \texttt{rdfs:subClassOf}.  To differentiate from Generalization-generated sub-classing, it is necessary to compare parent ontologies.  If the element declaring a sub-class has a parent ontology that is generated from a CDM, then there is no Realization possible and this must be a Generalization relationship.  However, for elements in LDM-generated ontologies, if the sub-class is contained in a CDM, then it is a Realization.  Similarly for elements in PDM-generated ontologies, if the sub-class is contained in an LDM, then it is a Realization.

This as a side-effect enables ``semantic polymorphism'': for example, by making a \texttt{logical:Measurement} a subclass of a \texttt{conceptual:Observable}, any data tagged with a specific measurement (e.g., ``50 Degrees Celsius'') is automatically inferred to be an instance of the concept ``Temperature''.  An application querying for ``Temperature'' will retrieve any temperature-related data regardless of whether it was recorded in Celsius, Fahrenheit, or Kelvin, provided the ontology links them correctly.    

\subsection{Observable}

UDDL Observables are mapped to \texttt{owl:Class}.  While \texttt{owl:DatatypeProperty} could be used, considering potential realizations of conceptual Observables as logical Measurements hints that a \texttt{owl:Class} is more flexible and appropriate mapping. 

\subsection{Entity}

The UDDL Entity element is mapped to a \texttt{owl:Class} in OWL as both represent a category of things sharing common features.

Observable Composition and Entity Composition will use the same mapping mechanism as both Observables and Entities are mapped to \texttt{owl:Class}.  Composition itself is modeled as an \texttt{owl:Restriction} with supporting \texttt{owl:ObjectProperty}s (e.g., \texttt{owl:hasComposition<Rolename>}) for the composed elements.

UDDL conceptual Entities define lower and upper bounds on compositions.  OWL Restrictions (\texttt{owl:Restriction}) handle this natively using the \texttt{owl:minCardinality} and \texttt{owl:maxCardinality} properties.  If the upper bound is $ -1 $ (unbounded), we simply omit the \texttt{owl:maxCardinality} restriction in OWL, as the OWA admits infinite cardinality by default.  We, of course, count instances of the composed elements when calculating multiplicities through \texttt{owl:onClass} within the \texttt{owl:Restriction}.

\subsection{Domain}

In UDDL, a Domain acts as a namespace or package containing Basis Entities.  We therefore map Domains to OWL Ontologies.

\subsection{Basis Entity}

In OWL, the distinction between a ``Basis'' entity and a standard ``Entity'' is primarily one of taxonomy, not metamodel.  Both function as Classes.  A Basis Entity acts as a root class in the domain ontology TBox (Terminological Box), while specialized Entities form the hierarchy below it.  Therefore, we map Basis Entities to \texttt{owl:Class} and specialized Entities to \texttt{owl:Class} using \texttt{rdfs:subClassOf}.

Note that Basis Entities may be contained within multiple Ontologies defined for UDDL Domains.

\subsection{Association}

The UDDL conceptual Association element is itself an Entity and therefore also mapped to a \texttt{owl:Class} in OWL using the same mapping mechanism for compositions.  Its \texttt{Participants} become two \texttt{owl:ObjectProperty}s, (e.g., \texttt{owl:hasParticipantSource<Rolename>} and \texttt{owl:hasParticipantTarget<Rolename>}).  These properties are assigned cardinality constraints based on the source and target multiplicities of the Participant.  Unlike Compositions, Participant Paths demand arbitrary traversal of OWL Object Properties.  In OWL 2, this concept is directly supported by Property Chain Axioms (\texttt{owl:propertyChainAxiom}).  This allows a property to be defined as the composition of several other properties.  Therefore, we map Participant Paths to two \texttt{owl:propertyChainAxiom}s (one in each direction) that assert the same sequence of nodes as in the Participant Path.  We will choose source and target properties based on the direction we are currently evaluating for a given Participant Path.

TODO: check Regularity Violations and how that compares to what you're allowed to have in Participant Paths cycles

\subsection{Query}

The UDDL query language is primarily a Selection and Projection language.  It traverses the graph through a \texttt{JOIN} mechanism and selects Observables.  This maps isomorphically to SPARQL, the standard query language for RDF.

\subsubsection{Select and Projection}

TODO: map UDDL \texttt{JOIN} to SPARQL traversal syntax.

% \begin{itemize}
%     \item \textbf{UDDL}: \texttt{SELECT rolename, characteristic}
%     \item \textbf{SPARQL}: \texttt{SELECT ?rolename ?characteristic}
% \end{itemize}

% \textbf{Path Traversal (The ``FROM'' Clause)}

% UDDL defines queries based on paths through the composition/participant hierarchy.
% \begin{itemize}
%     \item \textbf{UDDL}: \texttt{Entity.characteristic.subCharacteristic}
%     \item \textbf{SPARQL}: Property Paths (\texttt{/}). \texttt{?entity ex:hasCharacteristic/ex:hasSubCharacteristic ?value}.
% \end{itemize}

\subsubsection{Composite Queries (Unions)}

TODO:\texttt{UNION} clause supported in SPARQL.

% \begin{itemize}
%     \item \textbf{UDDL}: A Composite Query combines multiple queries, optionally as a Union.
%     \item \textbf{SPARQL}: The \texttt{UNION} clause.
% \end{itemize}

% \subsubsection{Example Translation:}
% UDDL Intent: Select the ``ID'' from both ``Aircraft'' entities and ``GroundVehicle'' entities (Union).

% \textbf{SPARQL:}
% \begin{verbatim}
% SELECT ?id
% WHERE {
%   {
%    ?s a ex:Aircraft ;
%        ex:hasID ?id.
%   }
%   UNION
%   {
%    ?s a ex:GroundVehicle ;
%        ex:hasID ?id.
%   }
% }
% \end{verbatim}

% In UDDL, a Query defines a View—a subset of data. In the Semantic Web, the concept of a ``View'' is often handled by SHACL (Shapes Constraint Language)~\cite{W3C_SHACL}.
% A SHACL Shape can define a ``closed'' shape that corresponds to the specific subset of properties selected by a UDDL View. This allows for validation: ensuring that a data packet conforms to the specific View definition required by an interface.

% TODO: need a way to exclude traversing specializes relationships
% TODO: calculate multiplicities over participant paths

% \subsection{Modeling Perspectives}

% The UDDL standard highlights three perspectives: Observation, Entity-Association, and Application~\cite{OpenGroup2023UDDL}:
% \begin{itemize}
%     \item Observation Perspective: Defines the ``Dictionary'' of Observables and Measurements.
%     \item Entity-Association Perspective: Defines the ``Grammar'' of the domain (Concepts).
%     \item Application Perspective: Defines the ``Usage'' (Views/Queries).
% \end{itemize}

% In SWT, this can be implemented via Ontology Imports:
% \begin{itemize}
%     \item \texttt{sdm-observables.owl}: Contains Observables and Units, reusable across domains.
%     \item \texttt{domain-specific.owl}: Imports \texttt{sdm-observables.owl}.  Defines Domain-specific Entities and Associations.
%     \item \texttt{app-interface.owl}: Imports \texttt{domain-specific.owl}. Defines SHACL shapes (Queries) for specific message/interface definition and validation.
% \end{itemize}

% This modularity mirrors the UDDL separation of concerns, preventing monolithic, unmanageable ontologies.

\section{Methods}

Convert Air System DSDM to an ontology using the proposed mappings.

Expressivity Profiling: Determine the Description Logic (DL) complexity class (e.g., $\mathcal{ALC}$, $\mathcal{SHOIN}$, $\mathcal{SROIQ}$).  This tells you the worst-case computational complexity for reasoning.

Consistency and Satisfiability Check: Use a reasoner (like HermiT or Pellet) to check for logical contradictions.  This reveals hidden contradictions.

Entailment Ratio: Count the explicit axioms vs. inferred axioms after running a reasoner.  This gives a measure of ``Hidden knowledge''.  If the inferred count is high, the ontology is ``powerful'', i.e., it generates new data for free.  If it is zero, might not be utilizing OWL features effectively.

\section{Results}

TODO

\section{Analysis}

TODO

\section{Discussion}

Can we add constraints on constructions of ontologies that force conformance with UDDL data models?


\section{Conclusion}

TODO

% Bibliography
\printbibliography

\end{document}
