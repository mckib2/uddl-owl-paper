% =================================================================
% INCOSE Conference LaTeX Template V1.2 (Release Date: November 4th, 2025)
% Copyright (c) 2025 INCOSE
% 
% This template is provided for use in preparing manuscripts for
% INCOSE conferences. You may use, modify, and 
% distribute this template for academic and professional purposes.
% 
% This template is provided "as is" without warranty of any kind.
% The author(s) disclaim all warranties, express or implied,
% including but not limited to warranties of merchantability and
% fitness for a particular purpose.

% =================================================================

\documentclass[11pt,letterpaper]{article} % Remove this line if using A4 size.
%\documentclass[11pt,a4]{article} % A4 is also accepted and is typically used for non-US submissions.

% ---------- Core layout ----------
\usepackage[
  letterpaper,
  left=0.6in,right=0.6in,top=0.6in,bottom=0.6in,
  headheight=85pt,headsep=-45pt
]{geometry}
\raggedbottom
\usepackage{graphicx}
\graphicspath{{./}{figures/}}
\usepackage{float}
\usepackage{amsmath}
\usepackage[table]{xcolor}
\usepackage{tikz}

% ---------- Captions & float spacing ----------
\usepackage{caption}
\captionsetup{
  font={sf,bf,footnotesize},
  labelfont={sf,bf,footnotesize},
  justification=centering,
  labelsep=period,
  hypcap=false
}
\setlength{\textfloatsep}{8pt}
\setlength{\floatsep}{6pt}
\setlength{\intextsep}{8pt}
\setlength{\abovecaptionskip}{12pt}
\setlength{\belowcaptionskip}{1pt}

% ---------- Tables / lists ----------
\usepackage{booktabs}
\usepackage{array}
\usepackage{tabularx}
\usepackage{enumitem}
\setlist{nosep}
\definecolor{tableheader}{HTML}{D9D9D9}
\newcolumntype{P}[1]{>{\sffamily\centering\arraybackslash}p{#1}}
\newcolumntype{Y}{>{\sffamily\centering\arraybackslash}X}
\newcolumntype{A}[1]{>{\raggedright\arraybackslash}p{#1}}
\newcolumntype{M}[1]{>{\raggedright\arraybackslash}m{#1}}

% ---------- Fonts ----------
\usepackage[T1]{fontenc}
\usepackage[utf8]{inputenc}
\usepackage{PTSerif}
\usepackage[scaled]{helvet}
\renewcommand{\sfdefault}{phv}
\newcommand{\headingfont}{\sffamily}

% ---------- Headings ----------
\usepackage{titlesec}
\setcounter{secnumdepth}{1}

% Heading 1
\titleformat{\section}
  {\headingfont\bfseries\raggedright\fontsize{18pt}{18pt}\selectfont}{}{0em}{}%{\headingfont\bfseries\raggedright\fontsize{18pt}{18pt}\selectfont}{}{0.75em}{}
% Heading 2
\titleformat{\subsection}
  {\headingfont\bfseries\raggedright\fontsize{15pt}{16pt}\selectfont}{}{0.75em}{}
% Heading 3
\titleformat{\subsubsection}
  {\headingfont\bfseries\raggedright\fontsize{12pt}{14pt}\selectfont}{}{0.75em}{}

% Heading spacing
\titlespacing*{\section}{0pt}{7pt}{6pt}
\titlespacing*{\subsection}{0pt}{7pt}{6pt}
\titlespacing*{\subsubsection}{0pt}{7pt}{6pt}

\newcommand{\miniheading}[1]{%
  \par\noindent{\headingfont\bfseries\fontsize{12pt}{14pt}\selectfont #1}\par\vspace{4pt}%
}

% ---------- Paragraphing ----------
\setlength{\parindent}{0pt}
\setlength{\parskip}{6pt plus 1pt minus 1pt}

% ---------- Page numbers ----------
\usepackage{fancyhdr}
\pagestyle{fancy}
\fancyhf{}
\fancyhfoffset[R]{18pt}
\setlength{\footskip}{18pt}
\fancyfoot[R]{\sffamily\bfseries\footnotesize \thepage}
\renewcommand{\headrulewidth}{0pt}
\renewcommand{\footrulewidth}{0pt}

% INCOSE logo
\fancypagestyle{firstpage}{
  \fancyhf{}
  \fancyhfoffset[R]{18pt}
  \fancyhead[R]{%
    \smash{\raisebox{0pt}[0pt][0pt]{
      \begingroup\setlength{\fboxsep}{20pt}
        \colorbox{white}{\includegraphics[height=0.6in]{template-images/incose-logo.jpg}}%
      \endgroup
    }}
  }
  \fancyfoot[R]{\sffamily\bfseries\footnotesize \thepage}
  \renewcommand{\headrulewidth}{0pt}
  \renewcommand{\footrulewidth}{0pt}
}
% ---------- Title Formatting ----------
\makeatletter
\@ifundefined{theauthor}{}{\let\theauthor\relax}
\makeatother
\usepackage{titling}
\pretitle{\headingfont\bfseries\fontsize{24pt}{26pt}\selectfont\raggedright}
\posttitle{\par\vspace{-.3in}}
\preauthor{}\postauthor{}
\author{\mbox{}}
\date{}
\setlength{\droptitle}{-3.2\baselineskip}

% ---------- Author cards ----------
\newcommand{\authorcard}[5]{%
  {\headingfont\bfseries\fontsize{12pt}{14pt}\selectfont #1}\par
  {\headingfont\bfseries\fontsize{12pt}{14pt}\selectfont #2}\par
  {\headingfont\bfseries\fontsize{12pt}{14pt}\selectfont #3}\par
  {\headingfont\bfseries\fontsize{12pt}{14pt}\selectfont #4}\par
  {\headingfont\bfseries\fontsize{12pt}{14pt}\selectfont #5}\par
}

% ---------- Biography photo placeholder and entry ----------

\makeatletter
\newcommand{\authorpic}[1]{%
    \includegraphics[width=0.6in,height=0.6in,keepaspectratio,clip]{#1}%
}
\makeatother

\newcommand{\authorbioentry}[3]{%
  \noindent\begin{tabular}{@{}m{0.5in} M{\dimexpr\columnwidth-0.5in\relax}@{}}
    \authorpic{#1} & \textbf{#2}\par #3
  \end{tabular}\par\medskip
}

% ---------- Safe figure include ----------
\makeatletter
\newcommand{\colfig}[2][]{%
  \IfFileExists{#2}{\includegraphics[width=\linewidth,#1]{#2}}{%
    \fbox{\parbox[b][1.5in][c]{\linewidth}{\centering \textit{Missing figure: }#2}}}%
}
\makeatother

% ---------- References: APA via biblatex/biber ----------
\let\theauthor\relax
\usepackage{csquotes}
\usepackage[style=apa,backend=biber]{biblatex}
\addbibresource{references.bib}

% ---------- Highlight callouts ----------
\usepackage{changepage}
\newenvironment{highlight}[1][0.25in]{%
  \begin{adjustwidth}{#1}{#1}\itshape}{\end{adjustwidth}}

% ---------- Two-column setup ----------
\usepackage{multicol}
\setlength{\columnsep}{18pt}

% ---------- Hyperlinks ----------
\usepackage[hidelinks]{hyperref}

%Additional packages
%\usepackage[utf8]{inputenc}
%\usepackage[T1]{fontenc}
\usepackage{graphicx}  % For including PNG figures
\usepackage{float}     % For better figure placement
\usepackage{amsmath}   % For mathematical equations
\usepackage{amsfonts}
\usepackage{amssymb}
\usepackage{hyperref}  % For hyperlinks
\usepackage{url}       % For URLs in bibliography
\usepackage{authblk}   % For author affiliations
\usepackage{longtable} % For multi-page tables
\usepackage{siunitx}   % For SI units


% =========================
% ===== Title & Authors ===
% =========================
\title{Aligning the Universal Domain Description Language (UDDL) with Web Ontology Language (OWL)}

\begin{document}
\maketitle
\thispagestyle{firstpage}

% ---- Authors ----
% ---- For the initial paper submission, do not include any author information. For the final paper submission, format author information as shown below. ------------------

\noindent
\begin{tabular*}{\textwidth}{@{\extracolsep{\fill}} A{0.32\textwidth} A{0.32\textwidth} A{0.32\textwidth}}
  \authorcard{Author One}{Organization}{Street Address}{City, Province, Postal}{author.one@email.com} &
  \authorcard{Author Two}{Organization}{Street Address}{City, Province, Postal}{author.two@email.com} &
  \authorcard{Author Three}{Organization}{Street Address}{City, Province, Postal}{author.three@email.com} \\
  \multicolumn{3}{@{}c@{}}{\rule{0pt}{0.9\baselineskip}} \\[-0.2\baselineskip]%\multicolumn{3}{@{}c@{}}{\rule{0pt}{0.9\baselineskip}} \\[-0.2\baselineskip]
  \authorcard{Author Four}{Organization}{Street Address}{City, Province, Postal}{author.four@email.com} &
  \authorcard{Author Five}{Organization}{Street Address}{City, Province, Postal}{author.five@email.com} &
  \authorcard{Author Six}{Organization}{Street Address}{City, Province, Postal}{author.six@email.com}%\authorcard{Author Six}{Organization}{Street Address}{City, Province, Postal}{author.six@email.com}
\end{tabular*}
\addvspace{.75in}

% ---- Two columns begin immediately after authors ----
\begin{multicols*}{2}
\raggedcolumns

% ---- Copyright ----
{\headingfont\bfseries\fontsize{8pt}{12pt}\selectfont
Copyright~\textcopyright~ \the\year{} by the author(s). Permission granted to INCOSE to publish and use.}
\\
% =========================
% ===== Abstract/Keywords =
% =========================
\phantomsection
\miniheading{Abstract}

The engineering of complex, software-intensive systems is undergoing a fundamental shift toward the Modular Open Systems Approach (MOSA), where machine-interpretable data is a prerequisite for system integration.  Despite strategic initiatives, a pervasive challenge remains: the inability to formally bind semantics to data syntax.  While the Universal Domain Description Language (UDDL)\textregistered{} provides a robust framework for documenting data meaning, it currently lacks interoperability with tooling ecosystems that provide automated reasoning capabilities to verify whether data and interface definitions are physically or operationally meaningful within a system architecture.

This paper proposes a methodology to bridge this gap by mapping UDDL models to Web Ontology Language (OWL), enabling automated semantic verification to detect interoperability gaps long before physical integration.  Our results show that this hybrid approach preserves the structural and semantic rigor of UDDL data modeling while enabling automated semantic verification.  We apply this approach to a representative spacecraft interface model, where UDDL definitions are transformed into OWL classes and axioms.  System elements are instantiated as named individuals to verify data production and usage against interface definition specifications, demonstrating that standard reasoning engines can detect semantic contradictions.  We conclude that bridging UDDL with OWL provides systems engineers with powerful new tools to ensure data interoperability and consistency.

\phantomsection
\subsubsection{Keywords}
Universal Domain Description Language, UDDL, Web Ontology Language, OWL, Semantic Verification, MOSA, Interoperability, Ontology Alignment, SPARQL

% =========================
% ===== Main Content ======
% =========================
\section{Introduction}
\label{section:introduction}

In systems engineering, rigorous definition of data and interfaces is not merely a documentation exercise, but a fundamental requirement for the composition, maintenance, and future reuse of complex systems~\parencite{fusco2020approach, madni2014system, Friedenthal2014, INCOSE2035}.  In traditional systems engineering, data dictionaries and interface control documents (ICDs) have served as the primary means of semantic capture~\parencite{Friedenthal2014, price2013semantic}.  However, these artifacts are often static in nature and reliant on human interpretation~\parencite{Friedenthal2014}.  The Universal Domain Description Language (UDDL)\textregistered{} Technical Specification recognizes that ``[adequately] describing domain data is a common problem in computer-based systems.  Typical approaches involve creating documentation to accompany artifacts \ldots in an attempt to capture the semantics of the data exchanged''~\parencite{OpenGroup2023UDDL}.

Systems engineering practice has shifted in the past decade to emphasize open systems approaches and adoption of intelligent systems, where machine-interpretable data is a prerequisite for modern sociotechnical system integration~\parencite{INCOSE2035}.  Strategic initiatives such as the Modular Open Systems Approach (MOSA) directives~\parencite{esper2019modular, zimmerman2019considerations, guertin2025sustained} also drive the need for semantic alignment, particularly for developing complex cyber-physical systems.  To support this, recent Department of Defense (DoD) directives emphasize the formalization of interface documentation through standards like Open Mission Systems (OMS)~\parencite{USAF_OMS}, the Future Airborne Capability Environment (FACE)\textregistered{}~\parencite{OpenGroup2017FACE}, and Sensor Open Systems Architecture (SOSA)\textregistered{}~\parencite{lane2024pyramid}.  However, current MOSA-aligned workflows rely on domain experts to manually judge the validity of semantic specifications.  This manual process is a significant bottleneck in large-scale system integration.

Traditional document-based approaches fail to capture data semantics due to their inability to precisely bind meaning to structure~\parencite{price2013semantic, Friedenthal2014}, often resulting in heterogeneous documentation across disparate systems and domains of operation~\parencite{mzili2025interoperability, noura2019interoperability}.  While UDDL was developed specifically to eliminate this ambiguity by creating formal conceptual building blocks for data interfaces, it remains under-studied in systems and interoperability literature.

While UDDL is effective at defining the contextual meaning of interface data, it lacks the native capability to determine if those semantics are operationally \textit{meaningful} or reasonable within a specific system architecture.  For example, UDDL can define a data field as a mass measurement from a temperature sensor -- a statement that is syntactically valid but semantically nonsensical in a spacecraft context.  Because UDDL cannot detect these domain-level mismatches or reason over instance-level values to ensure they are appropriate for the intended operational context, interface modelers and consumers of data interface specifications continue to rely on manual, subjective expert review to make judgments about semantic validity.

Semantic Web Technologies (SWTs), designed to enable machine-readability and logical inference, offer a potential solution to this limitation within UDDL ~\parencite{berners2023semantic}.  At the core of this stack are the Resource Description Framework (RDF) and the Web Ontology Language (OWL).  RDF provides a means to create graph-based data models where knowledge is encoded as triples (Subject, Predicate, Object), offering high flexibility and linkability~\parencite{W3C_RDF11}.  OWL extends RDF with formal logical semantics (Description Logics), enabling systems to define complex relationships, cardinality constraints, and classifications~\parencite{W3C_OWL2}.

Despite development and growing industry use of UDDL and its predecessor data modeling languages since 2013~\parencite{OpenGroup2013FACE}, to our knowledge, there is no existing literature showing interoperability, traceability, or mappings between UDDL data models and OWL or RDF.  The objective of this work is to develop an approach that enables UDDL domain-specific data models to interoperate with SWTs, leading to more efficient system integration. 

\section{Background} \label{sec:background}

\begin{table*}[t]
  \centering
  \renewcommand{\arraystretch}{1.3}
  \begin{tabularx}{\textwidth}{@{}l p{5.5cm} X@{}}
      \rowcolor{tableheader}
      \headingfont\bfseries UDDL Construct & \headingfont\bfseries Definition \& Role in Mapping & \headingfont\bfseries Example \\
      \midrule
      \textbf{Basis Entity} & Axiomatic domain concept. & \texttt{Place} (Axiomatic concept) \\
      \addlinespace[4pt]
      \textbf{Observable} & Unique measurable trait. & \texttt{Position} (Measurable trait) \\
      \addlinespace[4pt]
      \textbf{Entity} & Concept characterized by composed Observables and other Entities. & \texttt{Person} (Composes \texttt{Identifier}) \\
      \addlinespace[4pt]
      \textbf{Association} & Relation of two or more Participants. & \texttt{Ownership} (Relates \texttt{Person} and \texttt{Car}) \\
      \addlinespace[4pt]
      \textbf{Composition} & Traversable parent-child link. & \texttt{Car} $\xrightarrow{composes}$ \texttt{Wheels} \\
      \addlinespace[4pt]
      \textbf{Participant} & Role-based link in an Association. & \texttt{Teaching} $\xrightarrow{role: instructor}$ \texttt{Person} \\
      \addlinespace[4pt]
      \textbf{Generalization} & "is-a" hierarchy. & \texttt{FixedWingAircraft} specializes \texttt{Aircraft} \\
      \bottomrule
  \end{tabularx}
  \caption{Core UDDL Conceptual Meta-Model Elements and Relationships.  This table defines the fundamental constructs used to build a Conceptual Data Model (CDM).  It distinguishes between Basis Entities (axiomatic domain concepts) and Observables (measurable traits), providing the framework for Entities and Associations.  By formalizing these elements, UDDL allows systems engineers to define the precise meaning of data independent of its implementation.  CDM, Domain, Basis Entity, and Observable examples are referenced from \cite{sdm_v3_1_15}.  We also distinguish between traversable relationships (Composition and Participant), which define the paths used for data discovery and SPARQL query generation, and non-traversable relationships (Generalization), which establish class hierarchies and multi-level model refinements.  These relationships provide the formal constraints required to attach semantics to data interfaces.} \label{tab:cdm_elements}
\end{table*}

\subsection{UDDL} \label{section:introduction-uddl-overview}
Any system developed against the FACE and SOSA technical standards necessarily use UDDL data models for interface specification.  This paper will focus primarily on the Conceptual Data Model (CDM), as the meaning (i.e., the semantics) of the data are completely described by conceptual elements and their relationships.  We will draw upon lower-level Logical and Platform Data Model (LDM, PDM, respectively) concepts to justify design decisions to ensure the mappings are compatible with lower-level refinements.  Therefore, we will review only the conceptual meta-model elements and their relationships.

Table \ref{tab:cdm_elements} informally summarizes the construction of relevant conceptual meta-model elements.  All elements are assumed to have a globally unique name as well as an optional textual description field, enforced via Object Constraint Language (OCL) constraints.  A CDM is a top-level package that collects and organizes all conceptual model elements.  CDMs can be nested to create hierarchies or other organizational structures and multiple CDMs may be included in a single root-level Data Model.  A Domain collects a set of Basis Entities, relating to well understood concepts by practitioners within a particular domain~\parencite{OpenGroup2023UDDL}.  Basis Entities represent unique concepts that are not expressible simply through model structure, e.g., the concepts of \texttt{Minimum} and \texttt{Maximum}~\parencite{sdm_v3_1_15}.  Observables are characteristics that can be observed or measured, e.g., \texttt{Position}, \texttt{Orientation}, \texttt{Identifier}.  Notice the examples for Basis Entity and Observable in Table \ref{tab:cdm_elements}: a \texttt{Position} is something that can be measured and is therefore an Observable while \texttt{Place} is a concept and not meaningfully measurable and is therefore a Basis Entity (i.e., you cannot measure \texttt{Place} directly, but you could measure the \texttt{Position} of a \texttt{Place}, or the \texttt{Extent} of a \texttt{Place}).  Entities represent a domain concept in terms of their Observables and other composed conceptual Entities.  Associations are themselves Entities but which also include two or more Participants that can associate model paths terminating in either an Entity or Observable.  The Participants in the Association are characterized through Participant Paths which are a sequence of nodes that specify an arbitrary path through the network of related Entities and Associations and follow traversal rules, namely that Compositions can be traversed in the direction of the composition and Participants of Associations can be traversed bidirectionally.

Associations have the ability to both express relationships between participants as well as compose Observables and Entities themselves, meaning that properties of Associations can be used to express the semantics of data interfaces.

Table \ref{tab:cdm_elements} additionally summarizes how relationships between core UDDL Conceptual Meta-Model Elements are constructed.  Composition relationships include optional multiplicity specifications to indicate the minimum and maximum number of elements that can be composed (-1 indicates unbounded).  If not specified, the default is 1.  Similarly, Participants of Associations include optional source and target multiplicities that are applied when calculating the multiplicities along the Participant Paths.  Participant multiplicities are assumed to be unbounded if not specified.


\subsection{UDDL Query Construction} \label{section:query_construction}

The UDDL Query grammar and associated construction rules defines a SQL-like language for specification of data semantics~\parencite{OpenGroup2023UDDL}.  Query specifications can be constructed at all three levels of the UDDL meta-model.  The only differences in expression of query specifications at the various levels are the meta-types of Entities and Associations (CDM used for conceptual query specifications, LDM for logical query specifications, etc.) and qualifiers that may be used (e.g., in a \texttt{WHERE} clause).  \texttt{WHERE} clauses and other qualifiers specify data filtering criteria and side-effects on generated syntax when combined with a FACE Template specification~\parencite{OpenGroup2017FACE} but do not contribute to the semantics of the query.  We therefore need not consider mapping of \texttt{WHERE} clauses and other qualifiers, including sub-query specifications.  Entities and Associations will map cleanly across realizations between levels of the UDDL meta-model.  Considering this along with the fact that Generalization relationships must also be realized across refinement levels, we will without loss of generality consider only conceptual query specifications with \texttt{FROM}, \texttt{SELECT}, and \texttt{JOIN} query language elements.

\subsection{Semantic Web Technologies} \label{section:introduction-swts-overview}

The SWT stack provides a layered architecture for representing, querying, and reasoning over machine-interpretable knowledge. Each layer builds on the capabilities of the layers below it, enabling progressively richer forms of semantic expression and analysis. In the context of this work, SWTs provide the formal foundations needed to represent the semantics implied by interface definitions and to assess their consistency with domain expectations.  Table \ref{tab:swt_overview} summarizes the core technologies relevant to this mapping.

At the base of the stack, RDF provides a graph-based data model in which knowledge is represented as triples consisting of a subject, predicate, and object. RDF offers a flexible and extensible representation that is well suited to integrating heterogeneous data sources and expressing relationships between entities. When interface data is mapped into RDF, individual data values can be represented as nodes within a graph and linked to other entities through explicitly defined relationships, preserving the contextual structure implied by the interface semantics.

\begin{table}[H]
  \centering
  \renewcommand{\arraystretch}{1.5}
  \begin{tabularx}{\columnwidth}{@{}A{0.2\columnwidth}>{\raggedright\arraybackslash}X@{}}
      \rowcolor{tableheader}
      \headingfont\bfseries Technology & \headingfont\bfseries Description \\
      \addlinespace[4pt]
      RDF & The Resource Description Framework provides a graph-based data model where knowledge is represented as triples (Subject, Predicate, Object), serving as the foundation for data interchange. \\
      \addlinespace[4pt]
      OWL & The Web Ontology Language extends RDF with formal Description Logic semantics, enabling the definition of complex class hierarchies, restrictions, and inferential rules. \\
      \addlinespace[4pt]
      SPARQL & The standard query language for RDF, allowing for the retrieval and manipulation of data stored in RDF format.
  \end{tabularx}
  \caption{Overview of Semantic Web Technologies.  This table summarizes the layered architecture of the Semantic Web stack.}
  \label{tab:swt_overview}
\end{table}

Building on RDF, the OWL introduces formal semantics, enabling the definition of domain concepts, relationships, and constraints in a machine-interpretable form.  In particular, we use the Description Logic profile of OWL (OWL2DL), which represents a decidable subset of First-Order Logic (FOL).  OWL supports the specification of class hierarchies, property restrictions, and logical conditions that capture domain knowledge beyond what can be expressed through data structure alone.  These semantics enable automated reasoning, allowing implicit knowledge to be inferred from explicitly stated facts.  In the context of interface analysis, OWL provides a means to determine whether the semantics implied by an interface definition or its associated data are consistent with domain-level expectations.

SPARQL (SPARQL Protocol and RDF Query Language) complements RDF and OWL by providing a standardized query language for retrieving and manipulating RDF data. SPARQL enables selective access to graph data based on structural patterns and semantic relationships, supporting both exploratory analysis and downstream processing. In an interface-driven workflow, SPARQL can be used to extract relevant subsets of mapped interface data, identify entities participating in specific relationships, or retrieve inferred knowledge produced by reasoning processes.

\subsection{Closed-World and Open-World Assumptions} \label{section:theory-cwa-owa}

UDDL is used to construct domain-specific data models (DSDMs), functioning as a descriptive language to talk about all the data flowing through system interfaces.  Any concept not able to be captured using this data model definitionally cannot be understood by any interfaced component in the system if the interfaces are fully characterized using a common data model.  This is equivalent to a closed-world assumption (CWA).  SWTs generally adopt an open-world assumption (OWA), in which the absence of information does not imply falsity. Under OWA, a knowledge base is assumed to be inherently incomplete, and additional information may always exist elsewhere.

The distinction between CWA and OWA has important implications for how knowledge is represented and interpreted.  Closed-world reasoning enables definitive conclusions based on explicitly stated information, while open-world reasoning supports distributed and incremental knowledge integration.  As UDDL relies on a CWA while Semantic Web Technologies adopt an OWA, representing UDDL concepts using SWTs requires the introduction of closure axioms in the form of disjoint axioms and properties with restrictions.  These axioms explicitly restrict which properties or relationships are considered complete for a given class, thereby simulating the structural rigor and completeness guarantees of closed-world representations within an open-world reasoning framework.

\section{UDDL to OWL Mapping} \label{sec:uddl2owl_mapping}

The Semantic Web standards, primarily RDF and OWL, are based on directed binary graphs where knowledge is expressed as triples connecting a subject to an object via a predicate~\parencite{W3C_RDF11}.  While this simplicity facilitates scalability and graph traversal, it complicates the modeling of complex, high-dimensional systems~\parencite{antelmi2023survey}.  Many systems engineering domains are naturally described by hypergraphs, where a single relation (hyperedge) connects an arbitrary set of nodes.  UDDL requires treating participant paths of associations as first-class citizens when modeling, enabling the definition of semantic relationships between \textit{paths} rather than merely between Entities and Observables~\parencite{OpenGroup2023UDDL}, requiring a hypergraph representation.  As OWL does not natively support modeling of hypergraphs, we utilize the $ N $-ary Relation Pattern (reification) to transform complex relationships into distinct individuals within the ontology.  By combining this reification with OWL Property Chain Axioms, UDDL path-centric logic can be converted to the OWL2DL framework.

The following subsections describe how each conceptual UDDL meta-model element is mapped to its OWL counterpart.

\subsection{Containers}

UDDL provides packaging constructs to organize definitions.  We map Containers directly to \texttt{owl:Ontology} resources.  The hierarchy of UDDL models is preserved through \texttt{owl:imports} statements, where a child model imports the ontology generated from its parent or dependent models.  This approach maintains the modularity of the UDDL specification while ensuring that all definitions are available to the reasoner.

\subsection{Generalizations}

UDDL Generalizations are mapped directly to the \texttt{rdfs:subClassOf} property in OWL.  When a UDDL model specifies that element $ A $ \texttt{specializes} element $ B $, the transformation generates an OWL axiom declaring class $ A $ as a subclass of class $ B $.  This preserves the strict set-inclusion semantics of UDDL Generalization, ensuring that every instance of the specialized concept is also recognized as an instance of the general concept.  Additionally, the transformation defines top-level subsumption axioms, ensuring that all Entities eventually specialize a root \texttt{Entity} class, maintaining a consistent ontology structure.

\subsection{Observables}

UDDL Observables are mapped to the \texttt{owl:Class} construct.  Each Observable declared in the UDDL model becomes a subclass of a common root \texttt{Observable} class.  To capture the distinct nature of different Observables, the transformation defines an \texttt{owl:AllDisjointClasses} axiom covering the set of all defined Observables.  This prevents an individual from being inferred as belonging to multiple conflicting Observable types simultaneously, enforcing semantic precision.  While \texttt{owl:DatatypeProperty} could theoretically represent simple values, mapping Observables to classes provides greater extensibility for future refinement into logical measurements without altering the fundamental ontological structure. 

\subsection{Entities}

UDDL Entities are mapped to the \texttt{owl:Class} construct, consistent with their role as categorization mechanisms.  The conversion process automatically generates a global \texttt{composes} Object Property with a domain of \texttt{Entity} and a range defined as the union of \texttt{Entity} and \texttt{Observable} classes.  Composition relationships are then realized as \texttt{owl:Restriction} axioms on the Entity class.  This approach uses the shared \texttt{composes} property and distinguishes components via the \texttt{owl:onClass} constraint.  Multiplicities are strictly enforced using \texttt{owl:minQualifiedCardinality} and \texttt{owl:maxQualifiedCardinality}, with unbounded compositions represented by the omission of the maximum cardinality constraint.  To facilitate bidirectional graph traversal, an inverse property \texttt{isComposedBy} is also generated.

\subsection{Domains}

UDDL Domains are collections of Basis Entities that define a vocabulary for a specific subject area.  We map each UDDL Domain to a distinct \texttt{owl:Ontology}.  This separation ensures that domain concepts are modular and reusable.  Other ontologies can reference these domains via \texttt{owl:imports}, allowing for the composition of complex system models from standard domain definitions.

\subsection{Basis Entities}

UDDL Basis Entities represent the fundamental axiomatic concepts within a Domain.  A Basis Entity is mapped to an \texttt{owl:Class}, identical to the mapping for standard Entities.  While Basis Entities often serve as the roots of taxonomic trees (using \texttt{rdfs:subClassOf} for derived concepts), they possess no special meta-model distinction in OWL.  They are simply classes that provide the grounding for the domain vocabulary.

\subsection{Associations}

UDDL Associations are mapped to the \texttt{owl:Class} construct, inheriting from the root \texttt{Entity} class.  Its participants are realized using the global \texttt{associates} Object Property (and its inverse \texttt{isParticipantIn}).  However, unlike simple direct relationships, UDDL participants can be defined by complex paths traversing multiple entities.  The OWL transformation handles this by simplifying the class-level definition: the \texttt{associates} restriction on the Association class targets the type of the final node in the participant path.  The full semantic rigor of the path is pushed to the instance level, where specific property chains can be verified.  By asserting the full path structure on instances, the model supports validation via \texttt{owl:propertyChainAxiom}, ensuring that the data actually traverses the required structural path defined in the UDDL model.  As source and target bounds of UDDL Association participants do not define semantics, the bounds of individual nodes in the participant path are excluded from the mapping, instead using the effective bounds for the \texttt{associates} cardinality.


\begin{table*}[t]
  \centering
  \renewcommand{\arraystretch}{1.5}
  \begin{tabularx}{\textwidth}{@{}l >{\raggedright\arraybackslash}X >{\raggedright\arraybackslash}X@{}}
      \rowcolor{tableheader}
      \textbf{\headingfont UDDL Component} & \textbf{\headingfont Logic} & \textbf{\headingfont SPARQL Graph Representation} \\
      \addlinespace[4pt]
      \texttt{FROM Entity AS a} & Root starting point. & Root Variable $ ?a $ \\
      \addlinespace[4pt]
      \texttt{JOIN Target AS b ON a.role} & Edge traversal. & Triple Pattern $ (?a \text{ :role } ?b) $ \\
      \addlinespace[4pt]
      \texttt{SELECT b.attr} & Data projection. & Select Variable $ ?b\_attr $ from $ (?b \text{ :attr } ?b\_attr) $ \\
      \addlinespace[4pt]
      \texttt{AND} (in Joins) & Path convergence. & Shared Variable $ ?b $ used in multiple patterns (e.g., $ ?a \to ?b $ AND $ ?c \to ?b $) \\
      \bottomrule
  \end{tabularx}
  \caption{Mapping UDDL Query Components to Semantic Graph Patterns.  This table outlines the formal translation of UDDL query logic into the graph-based structures of SPARQL.  It bridges the gap between the relational style of UDDL (using tables and joins) and the triple-based model of Semantic Web Technologies (using nodes and edges).  By mapping SQL-like components to specific graph patterns, we enable automatic verification of the semantics of data interfaces, satisfying the precise topological and semantic requirements of the system architecture.}
  \label{table:ast_mapping}
\end{table*}

\subsection{Queries}

The UDDL Query grammar defines a SQL-like language for specifying data semantics through Selection and Projection.  It traverses the model graph through a series of \texttt{JOIN} operations and selects specific Observables.  To enable automated verification, we convert these specifications into SPARQL patterns using a two-step process: path resolution and graph pattern generation.  Table \ref{table:ast_mapping} summarizes how the primary components of a UDDL query (Selection, Roots, Joins, and Logic) map to the graph-based concepts of SPARQL.

For the path resolution step, the UDDL query specification is analyzed to trace the absolute navigational path from a root entity to every join alias.  This step resolves complex topological requirements, such as ``Diamond Joins,'' where an entity must satisfy multiple incoming relationships simultaneously.

For the graph pattern generation step, the resolved paths are transformed into SPARQL triple patterns.  Each UDDL alias becomes a SPARQL variable, and each relationship becomes a logic statement.  By using shared variables for converged paths, the engine enforces the precise structural and semantic requirements of the system architecture.

\section{Methods} \label{sec:methods}

We generated an example Interface Control Document (ICD) based on a generic satellite design (See \nameref{sec:Appendix}).  Based on this ICD, we created a UDDL data model using the PHENOM\textregistered{} modeling environment, and exported the model as XMI~\parencite{musen2015protege, phenom}.  The UDDL data model (summarized in Table \ref{tab:uddl_summary}) captured the interface semantics using the UDDL Query Language backed by a conceptual Entity/Association data model.

% Generated by generate_summary_stat_table.py
% Arguments: face_file=/home/nicholas/research/uddl-owl-paper/src/examples/incose_uddl2owl.face
\begingroup
    \centering
    \begin{table*}[t]
    \renewcommand{\arraystretch}{1.5}
    \begin{tabularx}{\textwidth}{@{} >{\sffamily\raggedright\arraybackslash}p{0.35\textwidth} >{\sffamily\raggedright\arraybackslash}X @{}}
        \rowcolor{tableheader}
        \headingfont\bfseries UDDL Data Model Statistics & \headingfont\bfseries Model Examples \\
        \addlinespace[4pt]
        % Left Column: Statistics
        \begin{tabular}[t]{@{}l r@{}}
            Number of Entities: & 1 \\
            Number of Associations: & 16 \\
            Total Compositions: & 47 \\
            Total Participants: & 33 \\
            Total Queries: & 4 \\
            Total Projected Characteristics: & 19 \\
            Avg Projections/Query: & 4.8 \\
            Avg Entities/Query: & 1.0 \\
            Avg Join Conditions/Query: & 7.5 \\
            Max Projections: & 7 \\
            Max Join Conditions: & 12
        \end{tabular}
        &
        % Right Column: Examples
        \textbf{Most Central Entity}: SolarPanel (Score: 12) \newline
        \textit{Compositions}: Identifier, Orientation, Position, OperationalState, HealthState, ElectricPotential \par\vspace{6pt}

        \textbf{Most Central Association}: SolarPanel\_Charges\_Battery (Score: 8) \newline
        \textit{Compositions}: Identifier, Power, ElectricCurrent, HealthState, Temperature, ElectricPotential \newline
        \textit{Participants}: charger: SolarPanel\newline\phantom{\textit{Participants}: }charged: Battery \par\vspace{6pt}

        \textbf{Observation Pattern}: TemperatureSensor\_Observe\_LouverFrame \newline
        \textit{Compositions}: Identifier \newline
        \textit{Participants}: observer: TemperatureSensor\newline\phantom{\textit{Participants}: }observed: Frame\_PartOf\_Louver $\to$ part \par\vspace{6pt}

        \textbf{Assembly Pattern}: Blade\_PartOf\_Louver \newline
        \textit{Compositions}: Identifier, HealthState \newline
        \textit{Participants}: assembly: Louver\newline\phantom{\textit{Participants}: }part: Blade
    \end{tabularx}
    \caption{Summary statistics of the UDDL Conceptual Data Model (CDM). The table presents the total counts of core modeling elements: Entities, Associations, Compositions, and Participants, as well as Query statistics including complexity metrics (averages and maximums for projections, entities, JOINs, and join conditions per query). It identifies the most central elements based on their structural connectivity and property density. Additionally, it highlights representative associations for the Observation pattern (\textit{Observe}) and the Assembly pattern (\textit{PartOf}) with their respective compositions and participants.}
    \label{tab:uddl_summary}
    \end{table*}
\endgroup

The UDDL data model contains examples of Generalizations, Entities with multiple compositions, Associations with simple and complex participant paths, as well as UDDL Query specifications.  The FACE Shared Data Model (SDM)~\parencite{sdm_v3_1_15} provides examples for Domain, Basis Entity, and Observable model elements.

The UDDL data model was transformed into an OWL ontology using the mapping methodology presented in the \nameref{sec:uddl2owl_mapping} Section.  The generated ontology was then loaded into Prot\'eg\'e with the HermiT reasoner version 1.4.3 configured~\parencite{glimm2014hermit}.  After loading the ontology, the reasoner was started.  The Prot\'eg\'e environment was used to visualize and execute SPARQL queries against the ontology.  The Snap SPARQL plugin was used to allow queries to be run over inferred elements of the ontology~\parencite{horridge2015snap}.

To verify the semantic validity, we created instances for system data observations.  An Observation OWL class was created to allow named individuals representing measurements of Observables in the context implied by the UDDL Query specifications.  We then converted all the UDDL Query specifications into SPARQL queries and executed them against the generated ontology and individuals.  UDDL Queries were then slightly modified to imply different semantics and run against the original set of individuals.

The UDDL to OWL conversion was implemented in Python by processing the OWL and UDDL data model XML representations.  The UDDL Query specification to SPARQL Query conversion was also implemented using Python by processing parsed UDDL Query specification abstract syntax trees and building the corresponding SPARQL expressions.

\section{Results} \label{sec:results}

Individuals in the generated ontology with properties matching expected semantics were queried and found using converted SPARQL queries.  Because we identify individuals based on semantics, we can make inferences about any individual Observations in the ontology to test if they align with the data an interface expects.

A summary of the data model generated from the example satellite ICD is shown in Table \ref{tab:uddl_summary}.  Common UDDL data modeling patterns were used to describe the relationships required to define semantics as presented in the ICD, e.g., the Observation pattern and Assembly pattern.  Table \ref{tab:uddl_summary} also shows examples of Associations using these patterns.

Table \ref{tab:ontology_sparql_summary} summarizes the ontology generated from the UDDL data model.  The HermiT reasoner verified that there were no logical contradictions.  Table \ref{tab:example_queries} shows an example of the results of the UDDL Query specification conversion to SPARQL queries against the generated ontology.  All SPARQL queries returned the expected sets of individuals indicating that the system contains the expected data observations based on the interfaces' specifications.  Any changes to the UDDL Query specifications resulted in SPARQL queries that returned no results indicating the expected semantic inconsistency (i.e., interfaces that require data that is not present in the system).


% Generated by generate_example_queries_table.py
% Arguments: face_file=/home/nicholas/research/uddl-owl-mapping/src/examples/incose_uddl2owl.face, entity=Satellite, max_depth=4
\begingroup
    \centering
    \begin{table*}[t]
    \renewcommand{\arraystretch}{1.2}
    \footnotesize
    \begin{tabularx}{\textwidth}{@{} >{\raggedright\arraybackslash}p{0.15\textwidth} >{\raggedright\arraybackslash}X >{\raggedright\arraybackslash}X @{}}
        \rowcolor{tableheader}
        \headingfont\bfseries Description & \headingfont\bfseries UDDL Query & \headingfont\bfseries SPARQL Query \\
        \addlinespace[4pt]
        Simple selection of Satellite attributes & \begin{minipage}{\linewidth}\raggedright\scriptsize \texttt{SELECT identifier} \\ \texttt{FROM Satellite} \end{minipage} & \begin{minipage}{\linewidth}\raggedright\scriptsize \texttt{SELECT ?obs\_0\_identifier} \\ \texttt{WHERE \{} \\ \texttt{?obs\_0\_identifier a :Observation .} \\ \texttt{?obs\_0\_identifier :hasObservable :Identifier .} \\ \texttt{?obs\_0\_identifier :hasSatellite :Satellite .} \\ \texttt{\}} \end{minipage} \\
        \addlinespace[2pt]
        Join through PartOf association to Louver & \begin{minipage}{\linewidth}\raggedright\scriptsize \texttt{SELECT identifier,} \\ \texttt{Louver.healthState} \\ \texttt{FROM Satellite} \\ \texttt{JOIN Louver\_PartOf\_Satellite} \\ \texttt{ON Louver\_PartOf\_Satellite.assembly = Satellite} \\ \texttt{JOIN Louver} \\ \texttt{ON Louver\_PartOf\_Satellite.part = Louver} \end{minipage} & \begin{minipage}{\linewidth}\raggedright\scriptsize \texttt{SELECT ?obs\_0\_identifier ?obs\_1\_healthState} \\ \texttt{WHERE \{} \\ \texttt{?obs\_0\_identifier a :Observation .} \\ \texttt{?obs\_0\_identifier :hasObservable :Identifier .} \\ \texttt{?obs\_0\_identifier :hasSatellite :Satellite .} \\ \texttt{?obs\_1\_healthState a :Observation .} \\ \texttt{?obs\_1\_healthState :hasObservable :HealthState .} \\ \texttt{...} \end{minipage} \\
        \addlinespace[2pt]
        Nested PartOf join to Frame & \begin{minipage}{\linewidth}\raggedright\scriptsize \texttt{SELECT identifier,} \\ \texttt{Louver.healthState,} \\ \texttt{Frame.identifier} \\ \texttt{FROM Satellite} \\ \texttt{JOIN Louver\_PartOf\_Satellite} \\ \texttt{ON Louver\_PartOf\_Satellite.assembly = Satellite} \\ \texttt{JOIN Louver} \\ \texttt{...} \end{minipage} & \begin{minipage}{\linewidth}\raggedright\scriptsize \texttt{SELECT ?obs\_0\_identifier ?obs\_1\_healthState ?obs\_2\_identifier} \\ \texttt{WHERE \{} \\ \texttt{?obs\_0\_identifier a :Observation .} \\ \texttt{?obs\_0\_identifier :hasObservable :Identifier .} \\ \texttt{?obs\_0\_identifier :hasSatellite :Satellite .} \\ \texttt{?obs\_1\_healthState a :Observation .} \\ \texttt{?obs\_1\_healthState :hasObservable :HealthState .} \\ \texttt{...} \end{minipage} \\
    \end{tabularx}
    \caption{Progressive complexity examples of UDDL queries for \texttt{Satellite} and their SPARQL translations. The queries demonstrate increasing complexity from simple attribute selection to multi-join queries traversing associations and compositions. SPARQL queries omit the \texttt{PREFIX} declaration for brevity.}
    \label{tab:example_queries}
    \end{table*}
\endgroup

% Generated by generate_ontology_summary_stat_table.py
% Arguments: face_file=/home/nicholas/research/uddl-owl-mapping/src/examples/incose_uddl2owl.face
\begingroup
    \centering
    \begin{table*}[t]
    \renewcommand{\arraystretch}{1.5}
    \begin{tabularx}{\textwidth}{@{} >{\sffamily\raggedright\arraybackslash}p{0.35\textwidth} >{\sffamily\raggedright\arraybackslash}X @{}}
        \rowcolor{tableheader}
        \headingfont\bfseries Ontology Statistics & \headingfont\bfseries SPARQL Query Statistics \\
        \addlinespace[4pt]
        % Left Column: Ontology Statistics
        \begin{tabular}[t]{@{}l r@{}}
            Classes: & 58 \\
            Object Properties: & 54 \\
            Individuals: & 19 \\
            Subclass Relations: & 40 \\
            Property Domains: & 3 \\
            Property Ranges: & 1 \\
            Inverse Properties: & 2 \\
            Disjoint Classes: & 55
        \end{tabular}
        &
        % Right Column: SPARQL Statistics
        \begin{tabular}[t]{@{}l r@{}}
            Total Queries: & 4 \\
            Total SELECT Variables: & 19 \\
            Total WHERE Triples: & 113
        \end{tabular} \par\vspace{6pt}

        \textbf{Most Used Property}: hasObservable \newline
        \textit{Details}: Domain: Observation; Range: Observable \par\vspace{6pt}

        \textbf{Query Complexity}: \newline
        \textit{Avg Variables/Query}: 4.8 \newline
        \textit{Avg Triples/Query}: 28.2 \newline
        \textit{Max Variables}: 7 \newline
        \textit{Max Triples}: 40
    \end{tabularx}
    \caption{Summary statistics of the generated OWL ontology and SPARQL queries from the UDDL Conceptual Data Model. The table presents counts of ontology elements (classes, properties, individuals, and relationships) and SPARQL query characteristics (variables and triples). It identifies the most used property and provides query complexity metrics.}
    \label{tab:ontology_sparql_summary}
    \end{table*}
\endgroup

\section{Analysis and Discussion} \label{sec:analysis}

The conversion process effectively addressed core challenges in representing UDDL data models within the OWL framework.  Our results show that by mapping UDDL Observables to disjoint OWL classes and applying closure axioms, reasoning engines can programmatically detect mismatches.  To resolve path-centric logic, we addressed UDDL's complex Participant Paths which require hypergraph representations not natively supported by OWL's binary structure.  The successful use of the $ N $-ary Relation pattern to reify paths allowed the SPARQL engine to satisfy convergence requirements (e.g. Diamond Joins) by binding multiple graph patterns to shared variables.  Individuals and generated property chain axioms were required for verification of interface semantics, as the structure of the Association was not fully representable using OWL class and Object Property constructs.  Any change to the SPARQL expressions resulted in the return of an empty set, confirming that the translation process preserves the rigor of the original UDDL specification.  This demonstrates that the system identifies only those data instances that satisfy the precise topological and semantic requirements of the interface.

The mapping of UDDL constructs to OWL enables a shift from manual, subjective interface review toward objective, machine-interpretable verification.  Automating these tests and verifications reduces the effort required to detect semantically incoherent definitions long before physical integration, reducing unplanned effort and interoperability gaps.  These reasoning capabilities are particularly valuable for large-scale systems with thousands or tens of thousands of complexly-related Entities and Associations, where manual review is infeasible.  For programs already aligned with FACE or SOSA that have semantically-enabled UDDL data models, these inference and verification capabilities may be immediately applicable using the ontology generation methodology proposed in this work.

An assumption in this work is that we may only consider conceptual UDDL data model content, whereas platform Queries based on realizations of conceptual Entities and Associations are used in practice during logical and physical system integration.  While multiple realizations of a single conceptual Entity or Association may be used at the platform level, the traceability of core semantics is ultimately anchored at the conceptual level and therefore the methods proposed in this work are still applicable through realization relationships.  Future work may explore data representation approaches that allow for the integration of logical and platform model content, e.g., through a mapping to XML Schema Datatypes compatible with RDF and OWL, enabling more comprehensive syntactic and semantic verification and integration strategies.

Mapping UDDL Domains and Domain-Specific Data Models (DSDMs) to distinct ontologies allows domain-specific vocabularies to become modular and reusable across different system models, e.g., for use with families of systems (FoS).  This aligns with strategic initiatives like MOSA to ensure semantic context is preserved across disparate systems.  The reuse of DSDMs across systems and programs further enhances interoperability and reduces the effort required to create and maintain domain-specific ontologies and semantic verification pipelines similar to those proposed in this work.

Another aspect of automated conversion is how to translate modeling patterns and ``idioms.''  UDDL data modeling patterns such as the Observation and Assembly pattern are useful for building and expressing data semantics, but are not usually aligned with standard OWL ontology construction.  More complex, idiomatic models and ontologies likely must be considered and aligned to gain insights about the ergonomics of constructs arising from automated translation.

The reverse mapping (OWL to UDDL) is not considered in this paper.  Most UDDL concepts have unambiguous mappings from UDDL to OWL, however, while UDDL is focused on the core use-case of data semantic specification, OWL allows for general constructs and relations that are not easily mappable to UDDL.  Future work should consider what restrictions on ontology construction must be enforced to guarantee two-way mappings without loss of information.

Automated ontology alignment techniques could further reduce the manual effort required to link reference system models with generated ontologies based on UDDL data models, moving closer to fully autonomous semantic integration and verification pipelines.  We expect that ontology alignment would be assisted by properly encoding complex path participants from Associations within the structure of the ontology which may require extensions to RDF and/or OWL (e.g., see ontological hypergraph representation approaches in \cite{hartig2014foundations, hu2013semantic, demir2010biopax, fabregat2018reactome}).   Fuzzy SPARQL query matching techniques could further reduce the manual effort required to match UDDL query specifications to SPARQL query patterns and allow more flexibility in specifying and identifying congruent interface semantic specifications.

\section{Conclusion} \label{sec:conclusion}

This paper has presented a formal mapping between UDDL and OWL to address the challenge of automatically assessing data ambiguity in complex system integration.  UDDL enables unambiguous documentation of data semantics while OWL provides the native capability to verify whether data interface definitions are operationally meaningful within a specific domain.  By transforming UDDL models into OWL classes and axioms and converting UDDL Query specifications into SPARQL patterns, we provide a framework for automated semantic verification, validated using a representative satellite interface model.

Practitioners using this methodology can move beyond human-in-the-loop interface reviews by integrating OWL-based reasoning into the systems engineering process.  Organizations can programmatically identify semantically incoherent interface definitions (e.g., mismatched sensor-to-sink logic) during the design phase.  This approach transforms the manual review bottleneck into a scalable, automated governance process.

The bridge between UDDL and OWL provides a mechanism to enforce a single technical source of truth across disparate system components.  While UDDL maintains the structure required for FACE or SOSA alignment, the OWL mapping provides formal rigor that provides evidence that data exchanged between a system's subsystems is logically consistent and operationally valid.

For programs adhering to a MOSA, the proposed methodology enables the creation of modular, reusable semantic libraries.  By mapping domain-specific UDDL models to distinct ontologies, systems engineers can verify the interoperability of new components against existing reference architectures with minimal manual effort, significantly reducing the risk and cost associated with long-term system evolution and technology insertion.


\section{Acknowledgments} \label{sec:acknowledgements}

This work was conducted using Prot\'eg\'e.

Google's Gemini 3 (Pro) was used to assist with scripting and \LaTeX\ typesetting.  All generated code was reviewed and tested for correctness.  All research, design, interpretation, and conclusions are original author work.

\newpage
\section{References}
\printbibliography[heading=none]

% ---------- Biography Format ----------
\newpage
\phantomsection
\makeatletter
\renewcommand{\authorbioentry}[3]{%
  \noindent\begin{tabular}{@{}m{0.5in} M{\dimexpr\columnwidth-0.5in\relax}@{}}
    \authorpic{#1} &
    {\headingfont\bfseries\raggedright\fontsize{12pt}{14pt}\selectfont #2}\par #3
  \end{tabular}\par\medskip
}
\makeatother
% ---------- Biography ----------
% ---- Note : Begin the Biography section on a new page

\section*{Biography}
\authorbioentry{template-images/author1_pic.jpg}{Author Name}{Provide a short biography of the author. Provide a short biography of the author.}
\authorbioentry{template-images/author2_pic.jpg}{Second Author}{Provide a short biography of the second author.}
\authorbioentry{template-images/author3_pic.jpg}{Third Author}{Provide a short biography of the third author.}


\newpage

\section{Appendix: Satellite Systems ICD} \label{sec:Appendix}

\subsection{Passive Thermal Control: Venetian-blind Louvers} \label{section:venetian_blind_louvers}

Louvers are mechanical components that passively aid the regulation of thermal convection between the satellite's internal components (payload, comm system, etc.) and the outer space vacuum. Louvers are usually employed on satellites because of their low mass, and their operation relies solely on temperature variation rather than generated power, leaving margin in the satellite power and mass budget.

\subsection{Location on Satellite and Systems Interactions} \label{section:louvers_location}

Louvers are mostly part of an external radiator assembly (being located on top of it), which is placed on the satellite flat panels or on deployable panels. Louvers could also be placed internally between surfaces to move the heat from the heat sources to radiators and ultimately away from the satellite. Or in general terms, they are located on top of a heat source while facing a cold sink (space vacuum).

\begin{table}[H]
    \centering
    \renewcommand{\arraystretch}{1.5}
    \begin{tabularx}{\columnwidth}{@{}P{0.45\columnwidth}Y@{}}
        \rowcolor{tableheader}
        \multicolumn{1}{c}{\headingfont\bfseries Interacting Systems} & \multicolumn{1}{c}{\headingfont\bfseries Primary Assignment} \\
        \addlinespace[4pt]
        Thermal Control & Thermal regulation. \\
        \addlinespace[4pt]
        Onboard Computer (OBC) & Temperature monitoring.
    \end{tabularx}
    \caption{Venetian-blind Louvers Interacting Systems}
    \label{tab:Venetian_blinds}
\end{table}


\subsection{Louver Sub-system Multiplicity} \label{section:louvers_multiplicity}

The satellite may include one or more louvers and each one of them operates individually and send its own telemetry messages.

\subsection{Louver Mechanical Assembly Components and Interfaces} \label{section:louvers_components}

Louvers are made of three main components:
\begin{itemize}
    \item The structural frame is mounted to a heat source within the satellite (internal or external).
    \item Louver blades are the elements that open and close, regulating heat convection and emissivity levels of the heat source of interest.
    \item Bimetallic Springs (made of two different aluminium alloys) are temperature-sensitive springs that connect the structural frame (in touch with the heat source) with the blades. They are bimetal to ensure that their bending will effectively open the blades at an angle. The spring materials have a set point temperature between around \SI{-20}{\celsius} and \SI{50}{\celsius} when the blades start to open. Then the blades go from fully closed to fully open within a \SI{10}{\celsius} to \SI{20}{\celsius} control band~\parencite{SierraSpace_Passive_Thermal_Louvers_2025}.
\end{itemize}

\begin{table}[H]
    \centering
    \renewcommand{\arraystretch}{1.5}
    \begin{tabularx}{\columnwidth}{@{}P{0.45\columnwidth}Y@{}}
        \rowcolor{tableheader}
        \multicolumn{1}{c}{\headingfont\bfseries Interface} & \multicolumn{1}{c}{\headingfont\bfseries Applicable to subsystem} \\
        \addlinespace[4pt]
        Telemetry & \checkmark \\
        \addlinespace[4pt]
        Command & $\times$ \\
        \addlinespace[4pt]
        Thermal & \checkmark \\
        \addlinespace[4pt]
        Mechanical & \checkmark \\
        \addlinespace[4pt]
        Power & $\times$
    \end{tabularx}
    \caption{Venetian-blind Louvers Interface Overview}
    \label{tab:louvers_interfaces}
\end{table}

\subsection{Mechanical Operation and Telemetry Messages} \label{section:louvers_operation}

During periods of higher operational needs (e.g. increase of data transmission between the communication subsystem and the ground station) the source of interest heats up, consequently increasing the temperature within the louvers' structural frame. The bimetallic springs, being installed on the frame, will consequently change state. As the springs receive heat, the metals that compose them will expand at different rates, which allows the blades to mechanically open. By rotating, they expose the heat source surface (high emissivity) to the cold sink so that more heat can be irradiated out into the space vacuum, without overheating the source and compromising critical operations.

When the heat source and the structural frame cool down, consequently, the springs will compress to close the blades, reducing heat dissipation (low-emissivity) and keeping the source warmer during low-demand periods.


\begin{table*}[t]
    \centering 
    \small
    \renewcommand{\arraystretch}{1.5}
    \begin{tabularx}{\textwidth}{@{}A{0.18\textwidth}A{0.16\textwidth}>{\raggedright\arraybackslash}XA{0.15\textwidth}A{0.15\textwidth}A{0.08\textwidth}@{}}
        \rowcolor{tableheader}
        \headingfont\bfseries Telemetry Message & \headingfont\bfseries Trigger & \headingfont\bfseries Notes & \headingfont\bfseries Sender & \headingfont\bfseries Receiver & \headingfont\bfseries Multiplicity \\
        \addlinespace[4pt]
        Louvers\_status\_info & Periodic \& out-of-range reach alert. & Temperature rise or drop outside the set range. Set range: \SI{-20}{\celsius} to \SI{50}{\celsius} & Louver assembly ID & Thermal Control \& OBC & 1 \\
        \addlinespace[4pt]
        Louvers\_health\_info & Periodic \& Mechanical fault alert. &  & Louver assembly ID & OBC & 1
    \end{tabularx}
    \caption{Venetian-blind Louvers Telemetry Message Overview}
    \label{tab:louvers_tm_overview}
\end{table*}


\begin{table*}[t]
    \centering
    \small
    \renewcommand{\arraystretch}{1.5}
    \begin{tabularx}{\textwidth}{@{}A{0.22\textwidth}>{\raggedright\arraybackslash}XA{0.10\textwidth}A{0.18\textwidth}A{0.15\textwidth}@{}}
        \rowcolor{tableheader}
        \headingfont\bfseries ID & \headingfont\bfseries Description & \headingfont\bfseries Data type & \headingfont\bfseries Units & \headingfont\bfseries Update Periodicity \\
        \addlinespace[4pt]
        Louver\_assembly\_id & Distinctive louver identifier. & Int &  & 1 Hz \\
        \addlinespace[4pt]
        Frame\_temperature & Measured through a temperature sensor mounted on the louver's structural frame. & Float & \SI{}{\celsius} & Every 1 second \\
        \addlinespace[4pt]
        Blade\_angle & Measured through a position sensor. & Float & \SI{}{\degree} & Every 1 second \\
        \addlinespace[4pt]
        Emissivity\_state & Derived parameter from the position sensor. & Enum & Low, Medium, High & Every 1 second
    \end{tabularx}
    \caption{Venetian-blind Louvers Status Information}
    \label{tab:louvers_status}
\end{table*}


\begin{table*}[t]
    \centering
    \small
    \renewcommand{\arraystretch}{1.5}
    \begin{tabularx}{\textwidth}{@{}A{0.22\textwidth}>{\raggedright\arraybackslash}XA{0.10\textwidth}A{0.18\textwidth}A{0.15\textwidth}@{}}
        \rowcolor{tableheader}
        \headingfont\bfseries ID & \headingfont\bfseries Description & \headingfont\bfseries Data type & \headingfont\bfseries Units & \headingfont\bfseries Update Periodicity \\
        \addlinespace[4pt]
        Louver\_assembly\_id & Distinctive louver identifier. & Int &  & 1 Hz \\
        \addlinespace[4pt]
        Health\_check & Overall structural integrity. & Enum & Nominal, Attention, Fault & Every 2 min \\
        \addlinespace[4pt]
        Blade\_mech\_compliance & The blades' smooth movement during operation. & Enum & Nominal, Attention, Fault & Every 2 min
    \end{tabularx}
    \caption{Venetian-blind Louvers Health Information}
    \label{tab:louvers_health}
\end{table*}

\subsection{Power Generation System: Solar Panels} \label{section:solar_panels}

A Solar Panel (also known as a photovoltaic cell or solar cell) is a flat surface device that directly transforms solar radiation into electrical power. Earth-orbiting satellites from LEO (Low Earth Orbit) to GEO (Geosynchronous Orbit) are the best candidates for solar panels as their power source.  Solar panels are used to convert sunlight into electrical power that the satellite systems and subsystems use to support their operations during the mission lifetime. The solar energy converted by the panels can also be stored in onboard batteries that will provide electrical energy to the satellite systems while in the Earth's shadow.

\subsection{Location on Satellite and Systems Interaction} \label{section:solar_panels_location}

Solar panels are placed on the satellite exterior, flat panels, or on deployable paddles that unfold once the satellite reaches its intended orbit. Their positioning is meant to maximize their sun exposure for uninterrupted power generation before passing into the Earth's shadow.

\begin{table}[H]
    \centering
    \renewcommand{\arraystretch}{1.5}
    \begin{tabularx}{\columnwidth}{@{}>{\sffamily\raggedright\arraybackslash}p{0.45\columnwidth}>{\sffamily\raggedright\arraybackslash}X@{}}
        \rowcolor{tableheader}
        \multicolumn{1}{c}{\headingfont\bfseries Interacting Systems} & \multicolumn{1}{c}{\headingfont\bfseries Primary assignment} \\
        \addlinespace[4pt]
        ADCS computer & Controls attitude to improve the Sun incidence on panels. \\
        \addlinespace[4pt]
        Onboard Computer (OBC) & Collect telemetry data, monitor power budgeting, and health of the panels. \\
        \addlinespace[4pt]
        Power Conditioning and Distribution Unit (PCDU) & Collect electricity generated by solar panels and distribute it to the satellite systems and the battery. \\
        \addlinespace[4pt]
        Mechanical Structure & Provides a stable structural support and alignment with the satellite reference frame.
    \end{tabularx}
    \caption{Solar Panels Interacting Systems}
    \label{tab:solar_interactions}
\end{table}

\subsection{Solar Panel Sub-system Multiplicity} \label{section:solar_multiplicity}

Multiple. There are usually more than one solar panel (fixed on a satellite panel or deployable) and each one of them operates individually and sends its own telemetry messages.

\subsection{Solar Panels Assembly Components and Interfaces} \label{section:solar_components}

\paragraph{Mechanical Components}
\begin{itemize}
    \item The deployable array/structural frame is a wing-like structure that connects multiple solar panels. During launch the deployable array are stowed and deploy once in orbit. It maintains panel pointing while surviving launch loads and on-orbit Sun incidence.
    \item The substrate is the panel backbone, is made of aluminium honeycomb core with composite or aluminum sheets, onto which the photovoltaic assembly is laced. It provides mechanical support, electrical insulation layers, and a flat surface to mount solar cells and interconnects.
    \item Coverglasses are thin radiation-resistant glass plates laced on top of each cell to protect against UV, ionizing radiation, and micrometeoroid/particle impacts. They also include optical coatings to reduce reflection.
    \item Interconnects are metallic conductors (ribbons or welds) that join cells into strings and connect strings to bus bars. Their geometry and material minimize resistive losses while allowing thermal expansion and mechanical flexibility.
    \item Encapsulant is the adhesive/potting material between cells, coverglass, and substrate that bonds and seals them from the space environment. It helps the stack tolerate wide temperature swings, outgassing limits, and radiation without embrittling or losing adhesion.
\end{itemize}
\paragraph{Electrical Components}
\begin{itemize}
    \item Solar cells are the photovoltaic devices that convert incident sunlight into DC electrical power. Cells are arranged into series/parallel to achieve the required array voltage and current for powering the satellite bus.
    \item Blocking diodes are placed in series with strings to prevent reverse current flow when parts of the array are shaded or damaged.
    \item The junction box is the interface node where strings come together, housing terminations, blocking/bypass diodes, and connectors to the satellite power provider~\parencite{barrett1971spacecraft}.
\end{itemize}

\begin{table}[H]
    \centering
    \renewcommand{\arraystretch}{1.5}
    \begin{tabularx}{\columnwidth}{@{}P{0.45\columnwidth}Y@{}}
        \rowcolor{tableheader}
        \multicolumn{1}{c}{\headingfont\bfseries Interface} & \multicolumn{1}{c}{\headingfont\bfseries Applicable to subsystem} \\
        \addlinespace[4pt]
        Telemetry & \checkmark \\
        \addlinespace[4pt]
        Command & \checkmark  \\
        \addlinespace[4pt]
        Thermal & \checkmark  \\
        \addlinespace[4pt]
        Mechanical & \checkmark  \\
        \addlinespace[4pt]
        Power/Electrical & \checkmark
    \end{tabularx}
    \caption{Solar Panels Interface Overview}
    \label{tab:solar_interfaces}
\end{table}

\subsection{Operation and Telemetry Messages}

Sun radiation reaches the solar cells at a determined angle, where photons excite semiconductor electrons and generate \SI{2}{\volt} per cell.  The electrical path stability is kept by the connection between Cells and the coverglass, with the substrate. The junction box collects the transformed solar energy and the panel strings terminate and combine to form the array output lines that feed the satellite power system. From the junction box, the generated power goes to the power management and distribution unit, which regulates the voltage, charges the batteries, and switches power to subsystems. During periods of sunlight, solar power runs the satellite and charges the batteries while during period of shadows, the batteries supply power to the bus~\parencite{reddy2008planning}.


\begin{table*}[t]
    \centering
    \small
    \renewcommand{\arraystretch}{1.5}
    \begin{tabularx}{\textwidth}{@{}A{0.22\textwidth}>{\raggedright\arraybackslash}XA{0.14\textwidth}A{0.12\textwidth}A{0.10\textwidth}@{}}
        \rowcolor{tableheader}
        \headingfont\bfseries Telemetry Message & \headingfont\bfseries Trigger & \headingfont\bfseries Sender & \headingfont\bfseries Receiver & \headingfont\bfseries Multiplicity \\
        \addlinespace[4pt]
        SolarPanel\_config\_state & Periodic \& mode change occurrence. & Solar Panel & OBC & 1 \\
        \addlinespace[4pt]
        SolarPanel\_health\_info & Periodic \& Mechanical/Electrical fault alert. & Solar Panel & OBC & 1
    \end{tabularx}
    \caption{Solar Panel Telemetry Message Overview}
    \label{tab:solar_panel_tm_overview}
\end{table*}

\begin{table*}[t]
    \centering
    \small
    \renewcommand{\arraystretch}{1.5}
    \begin{tabularx}{\textwidth}{@{}A{0.22\textwidth}>{\raggedright\arraybackslash}XA{0.08\textwidth}A{0.15\textwidth}A{0.18\textwidth}@{}}
        \rowcolor{tableheader}
        \headingfont\bfseries ID & \headingfont\bfseries Description & \headingfont\bfseries Data type & \headingfont\bfseries Units & \headingfont\bfseries Update Periodicity \\
        \addlinespace[4pt]
        SolarPanel\_id & Distinctive solar panel identifier. & Int &  & 1 Hz \\
        \addlinespace[4pt]
        SolarPanel\_deploy & Indicates deployment event. & Enum & Nominal, Attention, Fault & Periodic (at deployment), 1 per panel \\
        \addlinespace[4pt]
        SolarPanel\_temp & Measured panel temperature from integrated sensors. & Float & \SI{}{\celsius} & Periodic, 1 Hz per sensor \\
        \addlinespace[4pt]
        SolarPanel\_incid & Measures the angle of incidence between the sun and the panel. & Enum & \SI{}{\degree} & Periodic, 1 per panel \\
        \addlinespace[4pt]
        SolarPanel\_ops\_mode & Indicates how solar panels process power. & Enum & Off, safe, eclipse, fixed V, max power & Prompt by mode change \& 1 Hz
    \end{tabularx}
    \caption{Solar Panel Configuration State}
    \label{tab:solar_panel_configuration}
\end{table*}

\begin{table*}[t]
    \centering
    \small
    \renewcommand{\arraystretch}{1.5}
    \begin{tabularx}{\textwidth}{@{}A{0.22\textwidth}>{\raggedright\arraybackslash}XA{0.10\textwidth}A{0.18\textwidth}A{0.12\textwidth}@{}}
        \rowcolor{tableheader}
        \headingfont\bfseries ID & \headingfont\bfseries Description & \headingfont\bfseries Data type & \headingfont\bfseries Units & \headingfont\bfseries Update Periodicity \\
        \addlinespace[4pt]
        SolarPanel\_id & Distinctive solar panel identifier. & Enum &  &  \\
        \addlinespace[4pt]
        SolarPanel\_temperature & Solar array electronics internal temperature. & Float & \SI{}{\celsius} & 10 Hz \\
        \addlinespace[4pt]
        SolarPanel\_voltage\_in & Solar array voltage input. & Float & \SI{}{\volt} & 10 Hz \\
        \addlinespace[4pt]
        SolarPanel\_current\_out & Solar panel output current. & Float & \SI{}{\milli\ampere} & 10 Hz \\
        \addlinespace[4pt]
        SolarPanel\_tot\_power & Total power generated by solar array. & Float & \SI{}{\watt} &  \\
        \addlinespace[4pt]
        Health\_check & Internal health monitoring. & Enum & Nominal, Attention, Fault & 10 Hz \\
        \addlinespace[4pt]
        SolarPanel\_degradation & Power degradation. & Float & Nominal, Attention, Fault & 0.1 Hz
    \end{tabularx}
    \caption{Solar Panel Health Information}
    \label{tab:solar_panel_health}
\end{table*}

 
\end{multicols*}
\end{document}
